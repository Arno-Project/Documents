
\chapter{برنامه زمانی فازها}


برنامه‌ی زمانی مربوط به فاز فعلی و فازهای آینده‌ی پروژه به‌روز شده است. برای هر فاز بازه‌ی زمانی انجام تسک‌ها و همچنین وابستگی‌های اساسی آن‌ها در یک گانت چارت و مدت زمان تخمینی برای انجام هر تسک و یک نفر به عنوان مسئول در جدول زمان‌بندی آن فاز مشخص شده است. فرد مسئول لزوما همه‌ی کارهای تسک را به تنهایی انجام نمی‌دهد و بقیه‌ی افراد نیز مشارکت خواهند داشت ولی وی وظیفه‌ی پیگیری انجام آن تسک را برعهده دارد.

\section{تحلیل مقدماتی - تکرار اول}


\begin{table}[h]
	\centering
	\begin{tabular}{|p{0.07\linewidth}|p{0.35\linewidth}|p{0.1\linewidth}|p{0.15\linewidth}|p{0.15\linewidth}|p{0.07\linewidth}|} 
		
		\hline
		شناسه & نام وظیفه & مدت (روز) & شروع & پایان & پیش‌نیاز\\
		\hline
		\calendarEntry
		{1.1}
		{تخمین زمانی برای پروژه}
		{1}
		{2/30}
		{2/30}
		{}
		\calendarEntry
		{2.1}
		{تهیه قالب مستندات}
		{0.5}
		{2/31}
		{2/31}
		{}
		
		\calendarEntry
		{3.1}
		{تهیه لیست ریسک‌ها}
		{0.5}
		{2/31}
		{2/31}
		{1}
		
			\calendarEntry
		{4.1}
		{تهیه لیست نیازمندی‌ها}
		{0.5}
		{3/1}
		{3/1}
		{1}
		
			\calendarEntry
		{5.1}
		{اولویت بندی نیازمندی‌ها و ریسک‌ها}
		{0.5}
		{3/1}
		{3/1}
		{3,4}
		
		
			\calendarEntry
		{6.1}
		{رسم نمودار موارد کاربرد}
		{0.5}
		{3/2}
		{3/2}
		{5}
		
					\calendarEntry
		{7.1}
		{جزییات موارد کاربرد}
		{2}
		{3/2}
		{3/4}
		{6}
		
					\calendarEntry
		{8.1}
		{تکمیل واژه‌نامه و موارد باقی‌مانده مستندات}
		{0.5}
		{3/4}
		{3/4}
		{7}
		
		
	\end{tabular}
	\caption{زمان‌بندی تحلیل مقدماتی - تکرار اول}
\end{table}


\section{تحلیل تفصیلی - تکرار اول}


\begin{figure}[H]
	
	\begin{center}
		\begin{ganttchart}[
			expand chart=1\textwidth,
			vrule label font=\tiny,
			title label font=\tiny, 
			bar label font=\tiny, 
			y unit title=1cm,
			y unit chart=0.8cm,
			x unit=1cm,
			vgrid,hgrid, 
			title label anchor/.style={below=-1.6ex},
			title left shift=0,
			title right shift=0,
			title height=1,
			progress label text={},
			bar height=0.6,
			group right shift=0,
			group top shift=.5,
			group height=.2]{8}{15}

			%labels
			\gantttitle{خرداد}{8}
			\\
			\gantttitle{\rl{۸}}{1} 
			\gantttitle{\rl{۹}}{1} 
			\gantttitle{\rl{۱۰}}{1} 
			\gantttitle{\rl{۱۱}}{1} 
			\gantttitle{\rl{۱۲}}{1} 
			\gantttitle{\rl{۱۳}}{1} 
			\gantttitle{\rl{۱۴}}{1} 
			\gantttitle{\rl{۱۵}}{1} 
			%
			\\
			%tasks
			
			\ganttbar[progress=0, name=check]{\rl{استخراج چک‌لیست از اسلایدهای درس}}{8}{10} \\
			
			\ganttgroup{\rl{نیازمندی‌ها}}{8}{13} \\
			\ganttbar[progress=0, name=sched]{\rl{بروزرسانی زمان‌بندی}}{8}{9} \\
			\ganttgroup[name=update]{\rl{بروزرسانی محصولات}}{10}{13} \\
			\ganttbar[progress=0]{\rl{بروزرسانی نیازمندی‌ها}}{10}{12} \\
			\ganttbar[progress=0, name=risk]{\rl{بروزرسانی ریسک‌ها}}{10}{11} \\
			\ganttbar[progress=0, name=risk3]{\rl{ریسک‌های تکنیکی}}{11}{12} \\
			\ganttbar[progress=0, name=risk2]{\rl{اولویت‌بندی ریسک‌ها}}{12}{13} \\
			\ganttbar[progress=0]{\rl{بروزرسانی مواردکاربرد}}{10}{12} \\
			\ganttbar[progress=0, name=archReq]{\rl{نیازمندی‌های پراهمیت معماری}}{12}{12} \\
			\ganttbar[progress=0]{\rl{اضافه کردن نیازمندی‌های جدید}}{12}{13} \\

			\ganttbar[progress=0]{\rl{تکمیل موارد کاربرد}}{12}{13} \\
			
			\ganttgroup{\rl{تحلیل}}{12}{14} \\
			\ganttbar[progress=0, name=brain, bar right shift=-0.5]{\rl{\lr{brain storm} برای پیدا کردن کلاس‌ها}}{12}{12} \\
			\ganttbar[progress=0, name=crc, bar left shift=0.5]{\rl{کارت‌های CRC}}{12}{14} \\
			\ganttbar[progress=0, name=activity]{\rl{نمودارهای فعالیت}}{13}{14} \\			
			\ganttbar[progress=0, name=realization]{\rl{بررسی تحقق موارد کاربرد‌}}{14}{14} \\

			\ganttgroup[name=impl]{\rl{پیاده‌سازی}}{13}{15} \\
			\ganttbar[progress=0, name=back]{\rl{پیاده‌سازی بکند نمونه‌ اولیه سیستم }}{13}{14} \\
			\ganttbar[progress=0, name=front, bar left shift=0.5]{\rl{پیاده‌سازی فرانت‌اند نمونه اولیه‌ سیستم}}{14}{15} \\

			
			\ganttgroup{\rl{آزمون}}{15}{15} \\
			\ganttbar[progress=0, name=test]{\rl{ارزیابی دستاوردها}}{15}{15} \\			
			\ganttbar[progress=0]{\rl{تکمیل مستندات}}{13}{15} \\
			
			%relations 
			\ganttlink[link bulge=1.4, link mid=0.25]{risk}{risk2} 
			\ganttlink{risk3}{risk2} 
			\ganttlink[link mid=0.1]{check}{test} 
			\ganttlink{brain}{crc} 
			\ganttlink{crc}{activity}
			\ganttlink[link type=dr]{back}{front} 
			\ganttlink[link mid=0.4, link bulge=1.3]{archReq}{impl}
			\ganttlink{sched}{update}
			\ganttlink[link  bulge=0.2]{activity}{realization}
			\ganttlink[link mid = 0.9]{crc}{impl}
			
%			\ganttvrule[vrule/.append style={blue, thin},vrule offset=1]{\rl{گزارش اول}}{6}

		\end{ganttchart}
	\end{center}
	\caption{گانت چارت زمان‌بندی تحلیل تفصیلی - تکرار اول}
	
\end{figure}


\begin{table}[h]
	\centering
	\begin{tabular}{|p{0.07\linewidth}|p{0.45\linewidth}|p{0.13\linewidth}|p{0.25\linewidth}|} 
		
		\hline
		شناسه & نام وظیفه & مدت (روز) & مسئول\\
		\hline
			\calendarEntryWithName
		{1.2}
		{استخراج چک‌لیست از اسلایدهای درس}
		{1}
		{صبا هاشمی}
	\calendarEntryWithName
	{2.2}
	{بروزرسانی زمان‌بندی}
	{0.5}
	{صبا هاشمی}
	\calendarEntryWithName
	{3.2}
	{بروزرسانی نیازمندی‌ها با توجه به بازخوردها}
	{0.5}
	{علیرضا تاج‌میرریاحی}
	\calendarEntryWithName
	{4.2}
	{بروزرسانی ریسک‌ها با توجه به بازخوردها}
	{0.5}
	{مصطفی اوجاقی}

	\calendarEntryWithName
{5.2}
{استخراج ریسک‌های تکنینکی}
{0.5}
{امیرمهدی نامجو}


	\calendarEntryWithName
	{6.2}
	{اولویت‌بندی ریسک‌ها}
	{1}
	{مصطفی اوجاقی}
		\calendarEntryWithName
	{7.2}
	{بروزرسانی موارد کاربرد}
	{1}
	{امیرمهدی نامجو}
	\calendarEntryWithName
	{8.2}
	{استخراج نیازمندی‌های پراهمیت معماری}
	{1}
	{علیرضا تاج‌میرریاحی}
	\calendarEntryWithName
	{9.2}
	{اضافه کردن نیازمندی‌های جدید}
	{1}
	{امیرمهدی نامجو}
	
\calendarEntryWithName
{10.2}
{تکمیل موارد کاربرد}
{1}
{مصطفی اوجاقی}

\calendarEntryWithName
{11.2}
{\lr{brain storm} برای پیدا کردن کلاس‌ها
	}
{0.5}
{صبا هاشمی}
\calendarEntryWithName
{12.2}
{کارت‌های CRC}
{2}
{صبا هاشمی}
\calendarEntryWithName
{13.2}
{نمودارهای فعالیت}
{2}
{صبا هاشمی}
\calendarEntryWithName
{14.2}
{بررسی تحقق موارد کاربرد}
{0.5}
{علیرضا تاج‌میرریاحی}
\calendarEntryWithName
{15.2}
{پیاده‌سازی بکند نمونه اولیه‌ی سیستم}
{2}
{امیرمهدی نامجو}
\calendarEntryWithName
{16.2}
{پیاده‌سازی فرانت‌اند نمونه اولیه‌ی سیستم}
{1.5}
{علیرضا تاج‌میرریاحی}
\calendarEntryWithName
{17.2}
{ارزیابی دستاوردها}
{0.5}
{صبا هاشمی}
\calendarEntryWithName
{18.2}
{تکمیل مستندات}
{1}
{مصطفی اوجاقی}
		

	\end{tabular}
	\caption{زمان‌بندی تحلیل تفصیلی - تکرار اول}
\end{table}

\newpage

\section{تحلیل تفصیلی - تکرار دوم}


\begin{figure}[H]
	
	\begin{center}
		\begin{ganttchart}[
			expand chart=1\textwidth,
			vrule label font=\tiny,
			title label font=\tiny, 
			bar label font=\tiny, 
			y unit title=1cm,
			y unit chart=0.8cm,
			x unit=1cm,
			vgrid,hgrid, 
			title label anchor/.style={below=-1.6ex},
			title left shift=0,
			title right shift=0,
			title height=1,
			progress label text={},
			bar height=0.6,
			group right shift=0,
			group top shift=.5,
			group height=.2]{17}{26}
			
			%labels
			\gantttitle{خرداد}{10}
			\\
			\gantttitle{\rl{17}}{1} 
			\gantttitle{\rl{18}}{1} 
			\gantttitle{\rl{19}}{1} 
			\gantttitle{\rl{20}}{1} 
			\gantttitle{\rl{21}}{1} 
			\gantttitle{\rl{22}}{1} 
			\gantttitle{\rl{23}}{1} 
			\gantttitle{\rl{24}}{1} 
			\gantttitle{\rl{25}}{1} 
			\gantttitle{\rl{26}}{1} 
			%
			\\
			%tasks
			
			\ganttbar[progress=0, name=check]{\rl{استخراج چک‌لیست از اسلایدهای درس}}{17}{20} \\
			
			\ganttgroup{\rl{نیازمندی‌ها}}{17}{21} \\
			\ganttbar[progress=0, name=sched]{\rl{بروزرسانی زمان‌بندی}}{17}{17} \\
			\ganttbar[progress=0, name=update]{\rl{بروزرسانی و اصلاح محصولات نیازمندی تکرار قبل}}{18}{19} \\
			\ganttbar[progress=0, name=interface, bar right shift=-0.5]{\rl{تهیه‌ی واسط کاربری قابل اجرا‌}}{19}{21} \\
			\ganttbar[progress=0, name=interface2, bar left shift=0.5]{\rl{تهیه تصاویر از واسط کاربری قابل اجرا}}{21}{21} \\
			
			\ganttgroup[name=analyze]{\rl{تحلیل}}{20}{25} \\
			\ganttbar[progress=0, name=update2]{\rl{بروزرسانی و اصلاح محصولات تحلیل تکرار قبل}}{20}{20} \\
			\ganttbar[progress=0, name=pack]{\rl{بروزرسانی زیرسیستم‌ها و پیدا کردن بسته‌های سیستم‌}}{20}{20} \\						
			\ganttbar[progress=0, name=class1]{\rl{بروزرسانی کلاس‌ها‌}}{20}{20} \\						
			\ganttbar[progress=0, name=activity]{\rl{تهیه‌ی نمودارهای فعالیت با خطوط شنا}}{21}{22} \\
			\ganttbar[progress=0, name=seq]{\rl{تهیه‌ی نمودارهای توالی‌}}{21}{22} \\
			\ganttbar[progress=0, name=class2]{\rl{تهیه‌ی نمودار کلاس‌های تحلیل}}{23}{24} \\			
			\ganttbar[progress=0, name=realization]{\rl{بروزرسانی تحقق موارد کاربرد‌}}{24}{25} \\
			\ganttbar[progress=0, name=packDia]{\rl{تهیه‌ی نمودار بسته‌}}{24}{25} \\
			
		
						
			\ganttgroup{\rl{آزمون}}{25}{26} \\
			\ganttbar[progress=0, name=test, bar left shift=0.5]{\rl{ارزیابی دستاوردها}}{25}{26} \\			
			\ganttbar[progress=0]{\rl{تکمیل مستندات}}{25}{26} \\
			
			%relations 

			\ganttlink{interface}{interface2} 
			\ganttlink[link mid=0.1]{check}{test} 

			\ganttlink[link mid=0.75]{pack}{activity} 
			\ganttlink{class1}{activity} 						
			
			\ganttlink[link mid=0.5]{pack}{seq} 
			\ganttlink[link mid=0.25]{class1}{seq} 

			\ganttlink[link mid=0.75]{activity}{class2} 	
			\ganttlink{seq}{class2} 	

			\ganttlink[link mid=0.5]{class2}{realization} 				

			\ganttlink[link mid=0.25]{class2}{packDia} 						
			
			\ganttlink{update}{analyze}
			%			\ganttvrule[vrule/.append style={blue, thin},vrule offset=1]{\rl{گزارش اول}}{6}
			
		\end{ganttchart}
	\end{center}
	\caption{گانت چارت زمان‌بندی تحلیل تفصیلی - تکرار دوم}
\end{figure}

\begin{table}[h]
	\centering
	\begin{tabular}{|p{0.07\linewidth}|p{0.45\linewidth}|p{0.13\linewidth}|p{0.25\linewidth}|} 
		
		\hline
		شناسه & نام وظیفه & مدت (روز) & مسئول\\
		\hline
		\calendarEntryWithName
		{1.3}
		{استخراج چک‌لیست از اسلایدهای درس}
		{1}
		{صبا هاشمی}
		\calendarEntryWithName
		{2.3}
		{بروزرسانی زمان‌بندی}
		{0.5}
		{صبا هاشمی}
		\calendarEntryWithName
		{3.3}
		{بروزرسانی و اصلاح محصولات نیازمندی‌های تکرار قبل}
		{1}
		{امیرمهدی نامجو}
		\calendarEntryWithName
		{4.3}
		{تهیه‌ی واسط کاربر قابل اجرا}
		{2}
		{علیرضا تاج‌میرریاحی}
		
		\calendarEntryWithName
		{5.3}
		{تهیه تصاویر از واسط کاربری قابل اجرا}
		{0.5}
		{علیرضا تاج‌میرریاحی}
		
		
		\calendarEntryWithName
		{6.3}
		{بروزرسانی و اصلاح محصولات تحلیل تکرار قبل}
		{1}
		{مصطفی اوجاقی}
		\calendarEntryWithName
		{7.3}
		{بروزرسانی زیرسیستم‌ها و پیدا کردن بسته‌های سیستم}
		{0.5}
		{مصطفی اوجاقی}
		\calendarEntryWithName
		{8.3}
		{بروزرسانی کلاس‌ها}
		{0.5}
		{مصطفی اوجاقی}
		\calendarEntryWithName
		{9.3}
		{تهیه‌ی نمودارهای فعالیت با خطوط شنا}
		{3}
		{صبا هاشمی}
		
		\calendarEntryWithName
		{10.3}
		{تهیه نمودارهای توالی}
		{3}
		{امیرمهدی نامجو}
		\calendarEntryWithName
		{11.3}
		{تهیه‌ی نمودار کلاس‌های تحلیل}
		{2}
		{مصطفی اوجاقی}
		
			\calendarEntryWithName
		{12.3}
		{بروزرسانی تحقق موارد کاربرد}
		{2}
		{علیرضا تاج‌میرریاحی}
		
		\calendarEntryWithName
		{13.3}
		{تهیه نمودار بسته}
		{2}
		{امیرمهدی نامجو}

		\calendarEntryWithName
		{14.3}
		{ارزیابی دستاوردها}
		{1}
		{صبا هاشمی}
		\calendarEntryWithName
		{15.3}
		{تکمیل مستندات}
		{1}
		{علیرضا تاج‌میرریاحی}
		
		
	\end{tabular}
	\caption{زمان‌بندی تحلیل تفصیلی - تکرار دوم}
\end{table}
\newpage


\section{تحلیل تفصیلی - تکرار سوم}

\begin{figure}[H]
	
	\begin{center}
		\begin{ganttchart}[
			expand chart=1\textwidth,
			vrule label font=\tiny,
			title label font=\tiny, 
			bar label font=\tiny, 
			y unit title=1cm,
			y unit chart=0.8cm,
			x unit=1cm,
			vgrid,hgrid, 
			title label anchor/.style={below=-1.6ex},
			title left shift=0,
			title right shift=0,
			title height=1,
			progress label text={},
			bar height=0.6,
			group right shift=0,
			group top shift=.5,
			group height=.2]{-5}{25}
			
			%labels
			\gantttitle{خرداد}{6}
			\gantttitle{تیر}{25}
			\\
			\gantttitle{\rl{هفته‌ی آخر}}{6} 
			\gantttitle{\rl{هفته‌ی اول}}{7} 
			\gantttitle{\rl{هفته‌ی دوم}}{7} 
			\gantttitle{\rl{هفته‌ی سوم}}{7} 
			\gantttitle{\rl{هفته‌ی چهارم}}{4} 
			%
			\\
			%tasks
			
			\ganttbar[progress=0, name=check]{\rl{استخراج چک‌لیست از اسلایدهای درس}}{-5}{0} \\
			
			\ganttgroup{\rl{نیازمندی‌ها}}{-5}{-2} \\
			\ganttbar[progress=0, name=sched]{\rl{بروزرسانی زمان‌بندی}}{-5}{-5} \\
			\ganttbar[progress=0, name=update]{\rl{بروزرسانی و اصلاح محصولات نیازمندی تکرار قبل}}{-4}{-2}\\
	
			\ganttgroup[name=analyze]{\rl{تحلیل}}{-1}{1} \\
			\ganttbar[progress=0, name=update2]{\rl{بروزرسانی و اصلاح محصولات تحلیل تکرار قبل}}{-1}{1} \\
				
			\ganttgroup{\rl{طراحی‌}}{2}{15} \\			
			\ganttbar[progress=0, name=class]{\rl{تهیه‌ی نمودار کلاس‌های طراحی}}{2}{5} \\			
			\ganttbar[progress=0, name=comp]{\rl{تهیه‌ی نمودار مولفه‌}}{6}{9} \\
			\ganttbar[progress=0, name=seq]{\rl{تهیه‌ی نمودار توالی طراحی‌}}{10}{15} \\
			
			\ganttgroup{\rl{پیاده‌سازی}}{16}{22} \\			
			\ganttbar[progress=0, name=back]{\rl{تکمیل بکند پیاده‌سازی نسخه‌ی اولیه سیستم‌}}{16}{20} \\
			\ganttbar[progress=0, name=front]{\rl{تکمیل فرانت‌اند پیاده‌سازی نسخه‌ی اولیه سیستم‌}}{18}{22} \\			
			
			
			\ganttgroup{\rl{آزمون}}{23}{25} \\
			\ganttbar[progress=0, name=test, bar left shift=0.5]{\rl{ارزیابی دستاوردها}}{23}{24} \\			
			\ganttbar[progress=0]{\rl{تکمیل مستندات}}{23}{25} \\
			
			%relations 
			
			\ganttlink[link mid=0.1]{check}{test} 
			\ganttlink{back} {front}			
			\ganttlink{class} {comp}			
			\ganttlink{comp} {seq}			

			%			\ganttvrule[vrule/.append style={blue, thin},vrule offset=1]{\rl{گزارش اول}}{6}
			
		\end{ganttchart}
	\end{center}
	\caption{گانت چارت زمان‌بندی تحلیل تفصیلی - تکرار سوم}
\end{figure}

\begin{table}[h]
	\centering
	\begin{tabular}{|p{0.07\linewidth}|p{0.45\linewidth}|p{0.13\linewidth}|p{0.25\linewidth}|} 
		
		\hline
		شناسه & نام وظیفه & مدت (روز) & مسئول\\
		\hline
		\calendarEntryWithName
		{1.4}
		{استخراج چک‌لیست از اسلایدهای درس}
		{2}
		{صبا هاشمی}
		\calendarEntryWithName
		{2.4}
		{بروزرسانی زمان‌بندی}
		{0.5}
		{صبا هاشمی}
		\calendarEntryWithName
		{3.4}
		{بروزرسانی و اصلاح محصولات نیازمندی‌های تکرار قبل}
		{1}
		{امیرمهدی نامجو}
		\calendarEntryWithName
		{4.4}
		{بروزرسانی و اصلاح محصولات تحلیل تکرار قبل}
		{2}
		{مصطفی اوجاقی}
		\calendarEntryWithName
		{5.4}
		{تهیه‌ی نمودار کلاس‌های طراحی}
		{5}
		{مصطفی اوجاقی}
		\calendarEntryWithName
		{6.4}
		{تهیه‌ی نمودار مولفه}
		{3}
		{علیرضا تاج‌میرریاحی}
		
		\calendarEntryWithName
		{7.4}
		{تهیه نمودارهای توالی طراحی}
		{5}
		{امیرمهدی نامجو}
		
		\calendarEntryWithName
		{8.4}
		{تکمیل بکند پیاده‌سازی نسخه‌ی اولیه سیستم}
		{7}
		{امیرمهدی نامجو}
		
		\calendarEntryWithName
		{9.4}
		{تکمیل فرانت‌اند پیاده‌سازی نسخه‌ی اولیه سیستم}
		{7}
		{صبا هاشمی}
		
		\calendarEntryWithName
		{10.4}
		{ارزیابی دستاوردها}
		{2}
		{علیرضا تاج‌میرریاحی}
		\calendarEntryWithName
		{11.4}
		{تکمیل مستندات}
		{2}
		{مصطفی اوجاقی}
		
		
	\end{tabular}
	\caption{زمان‌بندی تحلیل تفصیلی - تکرار سوم}
\end{table}

\newpage
\section{ساخت - تکرار اول}


\begin{figure}[H]
	
	\begin{center}
		\begin{ganttchart}[
			expand chart=1\textwidth,
			vrule label font=\tiny,
			title label font=\tiny, 
			bar label font=\tiny, 
			y unit title=1cm,
			y unit chart=0.8cm,
			x unit=1cm,
			vgrid,hgrid, 
			title label anchor/.style={below=-1.6ex},
			title left shift=0,
			title right shift=0,
			title height=1,
			progress label text={},
			bar height=0.6,
			group right shift=0,
			group top shift=.5,
			group height=.2]{-6}{10}
			
			%labels
			\gantttitle{تیر}{7}
			\gantttitle{مرداد}{10}
			\\
			\gantttitle{\rl{هفته‌ی آخر}}{7} 
			\gantttitle{\rl{هفته‌ی اول}}{7} 
			\gantttitle{\rl{هفته‌ی دوم}}{3} 
			%
			\\
			%tasks
			
			\ganttbar[progress=0, name=check]{\rl{استخراج چک‌لیست از اسلایدهای درس}}{-6}{-5} \\
			
			\ganttgroup{\rl{نیازمندی‌ها}}{-5}{-4} \\
			\ganttbar[progress=0, name=sched]{\rl{بروزرسانی زمان‌بندی}}{-5}{-5} \\
			\ganttbar[progress=0, name=update]{\rl{بروزرسانی و اصلاح محصولات نیازمندی تکرار قبل}}{-4}{-4}\\
			
			\ganttgroup[name=analyze]{\rl{تحلیل}}{-3}{-2} \\
			\ganttbar[progress=0, name=update2]{\rl{بروزرسانی و اصلاح محصولات تحلیل تکرار قبل}}{-3}{-2} \\
			
			\ganttgroup{\rl{طراحی‌}}{-1}{5} \\			
			\ganttbar[progress=0, name=class]{\rl{بروزرسانی کلاس‌های طراحی}}{-1}{2} \\			
			\ganttbar[progress=0, name=seq]{\rl{بروزرسانی نمودارهای توالی طراحی‌}}{0}{3} \\
			\ganttbar[progress=0, name=olgo]{\rl{توضیح الگوهای استفاده شده‌}}{4}{5} \\
			\ganttbar[progress=0, name=db]{\rl{تهیه‌ی شمای پایگاه داده‌}}{4}{5} \\
			
			\ganttgroup{\rl{پیاده‌سازی}}{6}{10} \\			
			\ganttbar[progress=0, name=back]{\rl{بکند نمونه اولیه تکامل یافته سیستم‌}}{6}{9} \\
			\ganttbar[progress=0, name=front]{\rl{فرانت‌اند نمونه اولیه تکامل یافته سیستم‌}}{8}{10} \\			
			
			\ganttgroup{\rl{آزمون}}{10}{10} \\
			\ganttbar[progress=0, name=test]{\rl{ارزیابی دستاوردها}}{10}{10} \\			
			\ganttbar[progress=0]{\rl{تکمیل مستندات}}{10}{10} \\
			
			%relations 
			
			\ganttlink[link mid=0.1]{check}{test} 
	
			%			\ganttvrule[vrule/.append style={blue, thin},vrule offset=1]{\rl{گزارش اول}}{6}
			
		\end{ganttchart}
	\end{center}
	\caption{گانت چارت زمان‌بندی ساخت - تکرار اول}
\end{figure}

\begin{table}[h]
	\centering
	\begin{tabular}{|p{0.07\linewidth}|p{0.45\linewidth}|p{0.13\linewidth}|p{0.25\linewidth}|} 
		
		\hline
		شناسه & نام وظیفه & مدت (روز) & مسئول\\
		\hline
		\calendarEntryWithName
		{1.5}
		{استخراج چک‌لیست از اسلایدهای درس}
		{1}
		{صبا هاشمی}
		\calendarEntryWithName
		{2.5}
		{بروزرسانی زمان‌بندی}
		{0.5}
		{صبا هاشمی}
		\calendarEntryWithName
		{3.5}
		{بروزرسانی و اصلاح محصولات نیازمندی‌های تکرار قبل}
		{1}
		{امیرمهدی نامجو}
		\calendarEntryWithName
		{4.5}
		{بروزرسانی و اصلاح محصولات تحلیل تکرار قبل}
		{2}
		{مصطفی اوجاقی}
		\calendarEntryWithName
		{5.5}
		{بروزرسانی نمودار کلاس‌های طراحی}
		{3}
		{مصطفی اوجاقی}
		\calendarEntryWithName
		{6.5}
		{بروزرسانی نمودارهای تولی طراحی}
		{3}
		{امیرمهدی نامجو}
			\calendarEntryWithName
		{7.5}
		{توضیح الگوهای استفاده شده}
		{2}
		{علیرضا تاجمیر ریاحی}
		
			\calendarEntryWithName
		{8.5}
		{تهیه‌ی شمای پایگاه داده}
		{2}
		{صبا هاشمی}
		
		\calendarEntryWithName
		{9.5}
		{ بکند نمونه اولیه تکامل یافته سیستم}
		{5}
		{امیرمهدی نامجو}
		
		\calendarEntryWithName
		{10.5}
		{فرانت‌اند پیاده‌سازی نسخه‌ی اولیه سیستم}
		{5}
		{صبا هاشمی}
		
		\calendarEntryWithName
		{11.5}
		{ارزیابی دستاوردها}
		{2}
		{علیرضا تاج‌میرریاحی}
		\calendarEntryWithName
		{12.5}
		{تکمیل مستندات}
		{2}
		{مصطفی اوجاقی}
		
		
	\end{tabular}
	\caption{زمان‌بندی ساخت - تکرار اول}
\end{table}

\newpage

\section{ساخت - تکرار دوم}


\begin{figure}[H]
	
	\begin{center}
		\begin{ganttchart}[
			expand chart=1\textwidth,
			vrule label font=\tiny,
			title label font=\tiny, 
			bar label font=\tiny, 
			y unit title=1cm,
			y unit chart=0.8cm,
			x unit=1cm,
			vgrid,hgrid, 
			title label anchor/.style={below=-1.6ex},
			title left shift=0,
			title right shift=0,
			title height=1,
			progress label text={},
			bar height=0.6,
			group right shift=0,
			group top shift=.5,
			group height=.2]{10}{24}
			
			%labels
			\gantttitle{مرداد}{15}
			\\
			\gantttitle{\rl{10}}{1} 
			\gantttitle{\rl{11}}{1} 
			\gantttitle{\rl{12}}{1} 
			\gantttitle{\rl{13}}{1} 
			\gantttitle{\rl{14}}{1} 
			\gantttitle{\rl{15}}{1} 
			\gantttitle{\rl{16}}{1} 
			\gantttitle{\rl{17}}{1} 
			\gantttitle{\rl{18}}{1} 
			\gantttitle{\rl{19}}{1} 
			\gantttitle{\rl{20}}{1}
			\gantttitle{\rl{21}}{1} 
			\gantttitle{\rl{22}}{1} 
			\gantttitle{\rl{23}}{1} 
			\gantttitle{\rl{24}}{1} 
			
			%
			\\
			%tasks
			
			\ganttbar[progress=0, name=check]{\rl{استخراج چک‌لیست از اسلایدهای درس}}{10}{11} \\
			
			\ganttgroup{\rl{نیازمندی‌ها}}{10}{13} \\
			\ganttbar[progress=0, name=sched]{\rl{بروزرسانی زمان‌بندی}}{10}{10} \\
			\ganttbar[progress=0, name=update]{\rl{بروزرسانی و اصلاح محصولات نیازمندی تکرار قبل}}{11}{11}\\
			\ganttbar[progress=0, name=update]{\rl{بروزرسانی اولویت‌ها و ریسک‌ها}}{11}{12}\\
			\ganttbar[progress=0, name=update]{\rl{نسخه‌ی تکمیل شده‌ی بررسی تحقق موارد کاربرد}}{11}{13}\\
			
			\ganttgroup{\rl{طراحی‌}}{13}{14} \\			
			\ganttbar[progress=0, name=class]{\rl{تهیه‌ی نمودار استقرار}}{13}{14} \\			
			
			\ganttgroup{\rl{پیاده‌سازی}}{14}{23} \\			
			\ganttbar[progress=0, name=back]{\rl{بکند نسخه‌ی نهایی سیستم‌}}{14}{21} \\
			\ganttbar[progress=0, name=front]{\rl{فرانت‌اند نسخه‌ی نهایی سیستم‌}}{15}{22} \\			
						\ganttbar[progress=0, name=doc1]{\rl{تهیه‌ی مستند نصب سیستم‌}}{22}{23} \\			
			\ganttbar[progress=0, name=doc2]{\rl{تهیه‌ی مستند استفاده از سیستم‌}}{22}{23} \\			

			\ganttgroup{\rl{آزمون}}{23}{24} \\
			\ganttbar[progress=0, name=test]{\rl{ارزیابی دستاوردها}}{23}{24} \\			
						\ganttbar[progress=0, name=test]{\rl{درستی‌سنجی }}{23}{24} \\		
			\ganttbar[progress=0]{\rl{تکمیل مستندات}}{24}{24} \\
			
			%relations 
			
			\ganttlink[link mid=0.1]{check}{test} 
			\ganttlink[link mid=0.75]{back}{doc1} 
			\ganttlink[link mid=0.5]{back}{doc2}			
			\ganttlink[link mid=0.5]{front}{doc1} 
			\ganttlink[link mid=0.25]{front}{doc2}
			%			\ganttvrule[vrule/.append style={blue, thin},vrule offset=1]{\rl{گزارش اول}}{6}
			
		\end{ganttchart}
	\end{center}
	\caption{گانت چارت زمان‌بندی ساخت - تکرار دوم}
\end{figure}

\begin{table}[h]
	\centering
	\begin{tabular}{|p{0.07\linewidth}|p{0.45\linewidth}|p{0.13\linewidth}|p{0.25\linewidth}|} 
		
		\hline
		شناسه & نام وظیفه & مدت (روز) & مسئول\\
		\hline
		\calendarEntryWithName
		{1.6}
		{استخراج چک‌لیست از اسلایدهای درس}
		{1}
		{صبا هاشمی}
		\calendarEntryWithName
		{2.6}
		{بروزرسانی زمان‌بندی}
		{0.5}
		{صبا هاشمی}
		\calendarEntryWithName
		{3.6}
		{بروزرسانی و اصلاح محصولات نیازمندی‌های تکرار قبل}
		{1}
		{امیرمهدی نامجو}
		\calendarEntryWithName
		{4.6}
		{بروزرسانی اولویت‌ها و ریسک‌ها}
		{2}
		{مصطفی اوجاقی}
		\calendarEntryWithName
		{5.6}
		{نسخه‌ی تکمیل شده‌ی بررسی تحقق موارد کاربرد}
		{2}
			{علیرضا تاجمیر ریاحی}
	\calendarEntryWithName
		{6.6}
		{تهیه‌ی نمودار استقرار}
		{2}
		{مصطفی اوجاقی}
		
		\calendarEntryWithName
		{7.6}
		{ بکند نسخه‌ی نهایی سیستم}
		{5}
		{امیرمهدی نامجو}
		
		\calendarEntryWithName
		{8.6}
		{فرانت‌اند نسخه‌ی نهایی سیستم}
		{5}
			{علیرضا تاجمیر ریاحی}
		\calendarEntryWithName
		{9.6}
		{تهیه‌ی مستند نصب سیستم}
		{1}
		{صبا هاشمی}
		\calendarEntryWithName
		{10.6}
		{تهیه‌ی مستند استفاده از  سیستم}
		{1}
	{صبا هاشمی}
		
		\calendarEntryWithName
		{11.6}
		{ارزیابی دستاوردها}
		{1}
		{علیرضا تاج‌میرریاحی}
				\calendarEntryWithName
		{12.6}
		{درستی‌سنجی}
		{2}
		{امیرمهدی نامجو}
		\calendarEntryWithName
		{13.6}
		{تکمیل مستندات}
		{2}
		{مصطفی اوجاقی}
		
		
	\end{tabular}
	\caption{زمان‌بندی ساخت - تکرار دوم}
\end{table}


\newpage

\section{گذار - تکرار اول}

\begin{table}[h]
	\centering
	\begin{tabular}{|p{0.07\linewidth}|p{0.35\linewidth}|p{0.1\linewidth}|p{0.15\linewidth}|p{0.15\linewidth}|} 
		
		\hline
		شناسه & نام وظیفه & مدت (روز) & شروع & پایان\\
		\hline
		\calendarEntryWithoutDep
		{49}
		{تحویل نهایی}
		{1}
		{5/25}
		{5/25}
		
		
		
	\end{tabular}

	\caption{زمان‌بندی گذار - تکرار اول}
\end{table}

