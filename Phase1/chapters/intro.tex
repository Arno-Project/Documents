
\chapter{مقدمات}

\section{شرح سیستم آرنو}

سیستم مورد بررسی در این پروژه، ‌یک سیستم جامع اطلاعاتی برای مدیریت ارائه‌ی خدمات در خانه است. 
در این مسئله امکان ارائه‌ی خدمات در سه حوزه‌ی زیر مورد اهمیت است:

\begin{enumerate}
\item
نظافت: شامل نظافت بخش‌های مختلف منزل 
\item
حمل و نقل: شامل خدمات انتقال بسته و اسباب‌کشی
\item 
کارهای فنی: شامل خدمات لوله‌کشی و تاسیسات، برق‌کاری، تعمیرات لوازم خانگی، تعمیر رایانه و موبایل و تعمیر و سرویس خودرو

\end{enumerate}

افراد در تعامل با این سامانه، از یک سو مشتریانی بوده که درخواست ارائه‌ی خدمات را به سیستم می‌دهند و از سوی دیگر متخصصانی هستند که هر کدام می‌توانند در حوزه‌ی تخصصی خود به ارائه‌ی خدمت به مشتریان بپردازند. نوع خدمات ارائه شده در شرکت و توسط متخصصین مختلف می‌تواند در طول زمان دچار تغییر شود و این تغییرات در این سامانه پیش‌بینی شده است. همچنین یک ویژگی مورد توجه دیگر در این سامانه امکان ارزیابی متخصصین و مشتریان توسط یک‌دیگر است تا امکان شناسایی نقاط ضعف و قوت ارائه‌ی خدمات وجود داشته باشد.

به غیر از کاربران عادی، مخاطبان دیگر این سامانه مدیران شرکت هستند. مدیران کسب‌وکاری نیاز به تعامل با این سیستم جهت گرفتن گزارشات آماری و تفصیلی دارند تا روند ارائه‌ی خدمات را بررسی کند و برای جهت کسب‌و‌کار تصمیم بگیرند. مدیران فنی نیز گزارشات مشکلات فنی سیستم را دریافت و بررسی می‌کنند تا مشکلات پیش آمده را برطرف و نرخ خطاهای سیستم را کاهش دهند و درنتیجه رضایت‌مندی کاربران در استفاده از سیستم فراهم کنند.


\section{شرح مستندات}


