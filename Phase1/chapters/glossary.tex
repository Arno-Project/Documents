

\chapter{واژه‌نامه}

واژه‌نامه‌ی تولید شده شامل مجموعه‌ای از واژه‌های دنیای کسب و کار مسئله‌ی مد نظر است. کلمات استفاده شده در مدل‌ها به عنوان واژه‌ی اصلی آورده شده، و کلماتی که ممکن است به همان معنی استفاده شوند به عنوان مترادف کلمه ذکر شده‌اند. 

\newpage
\begin{center}
	\begin{singlespace}	\begin{longtable}{|m{0.18\linewidth}|m{0.82\linewidth}|} 
			\hline
			اصطلاح & تعریف\\
			\hline
			\glossaryTitle{ت}
			\glossaryEntry{تخصص}{مهارت انجام یک کار. هر خدمت درخواست شده نیازمند یک تخصص خاص و متخصصین دارای آن تخصص مهارت انجام آن خدمت را دارند.}{نوع خدمت}{-}
			\glossaryTitle{خ}
			\glossaryEntry{خدمت}{فعالیتی که مشتری به دنبال انجام آن است و توسط متخصصین سامانه ارائه می‌شود.}{سرویس، وظیفه}{-}
			\glossaryTitle{س}
			
			\glossaryEntry{سامانه}{سیستم  یا سامانه مجموعه‌ای از اجزای مختلف است که معمولاً برای رسیدن به یک هدف با یکدیگر در ارتباط و تعامل هستند. هر سیستم دارای ورودی، خروجی و پردازش مشخص است.}{سیستم}{-}
			\glossaryTitle{ک}
			\glossaryEntry{کاربر}{همه‌ی افرادی که از سیستم استفاده می‌کنند.}{-}{-}
			\glossaryEntry{کاربر عادی}{کاربرانی از سامانه که دسترسی مدیریتی ندارند؛ یعنی متخصصان و مشتریان.}{-}{-}
			\glossaryTitle{م}
			\glossaryEntry{متخصص}{نقشی در سیستم که مهارت انجام یک کار را دارد و می‌تواند خدمت‌های درخواست شده‌ی مربوط به تخصص خود را انجام دهد.}{-}{-}
			\glossaryEntry{مشتری}{‌نقشی در سیستم که درخواست انجام یک خدمت را در سیستم ثبت می‌کند.}{-}{-}
			\glossaryEntry{مدیر شرکت}{‌نقشی که دانش کسب‌و‌کار دارد، به تمام قسمت‌های سامانه دسترسی داشته و اجازه‌ی مدیریت کاربران و خدمات را دارد.}{-}{-}
			\glossaryEntry{مدیر فنی شرکت}{‌نقشی که به مشکلات فنی که در سیستم رخ می‌دهد رسیدگی می‌کند.}{-}{-}
		\end{longtable}
	\end{singlespace}
\end{center}