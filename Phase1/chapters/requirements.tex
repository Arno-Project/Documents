\chapter{نیازمندی‌ها}



\section{فهرست اولویت‌بندی‌شده‌ی نیازمندی‌ها}


 برای اولویت‌بندی نیازمندی‌ها ابتدا معیار 
\lr{MoSCoW}
و پس از آن میزان ریسک نیازمندی برای پروژه در نظر گرفته شده است. 
در ادامه نیازمندی‌های سیستم به همراه میزان اهمیت آن‌ها، ریسک، و هزینه‌ی تخمین‌زده شده از پیاده‌سازی آن‌ها به تفکیک زیرسیستم آورده شده است.

\subsection{زیرسیستم کاربری}

\begin{enumerate}
	\item
	مدیر شرکت باید امکان تایید متخصصان را داشته باشد.
	\\
	\textbf{اهمیت:} \lr{Must Have}
	\\
	\textbf{ریسک:} متوسط
	\\
	\textbf{هزینه‌ی پیاده‌سازی:} پایین
	\item 
	مدیر شرکت باید بتواند کاربران را بر اساس اطلاعات مختلف جست‌وجو کند.
	\\
	\textbf{اهمیت:} \lr{Must Have}
	\\
	\textbf{ریسک:} متوسط
	\\
	\textbf{هزینه‌ی پیاده‌سازی:} متوسط
	\item 
	مدیر شرکت باید بتواند امکان فعالیت یک متخصص را محدود کند.
	\\
	\textbf{\textbf{اهمیت:}} \lr{Must Have}
	\\
	\textbf{ریسک:} متوسط
	\\
	\textbf{هزینه‌ی پیاده‌سازی:} متوسط
	
	
	\item
	مدیر شرکت باید بتواند مشخصات همه‌ی کاربران را مشاهده کند.
	\\
	\textbf{اهمیت:} \lr{Must Have}
	\\
	\textbf{ریسک:} پایین
	\\
	\textbf{هزینه‌ی پیاده‌سازی:} متوسط
	
\item 
مشتریان و متخصصان باید امکان ثبت‌نام در سیستم را داشته باشند.
 \\
\textbf{اهمیت:} \lr{Must Have}
\\
\textbf{ریسک:} پایین
 \\
\textbf{هزینه‌ی پیاده‌سازی:} پایین
\item
همه‌ی کاربران سامانه باید امکان ورود به سیستم را داشته باشند.
 \\
\textbf{اهمیت:} \lr{Must Have}
\\
\textbf{ریسک:} پایین
\\
\textbf{هزینه‌ی پیاده‌سازی:} پایین

\item 
متخصص باید بتواند حساب کاربری خود را غیرفعال کند.
\\
\textbf{اهمیت:} \lr{Should Have}
\\
\textbf{ریسک:} متوسط
\\
\textbf{هزینه‌ی پیاده‌سازی:} پایین

\item 
مدیر شرکت باید بتواند مدیران دیگری به سیستم اضافه کند.
\\
\textbf{اهمیت:} \lr{Should Have}
\\
\textbf{ریسک:} پایین
\\
\textbf{هزینه‌ی پیاده‌سازی:} پایین

\item 
مدیر شرکت باید بتواند امکان فعالیت یک مشتری را محدود کند.
\\
\textbf{اهمیت:} \lr{Should Have}
\\
\textbf{ریسک:} پایین
\\
\textbf{هزینه‌ی پیاده‌سازی:} متوسط
\item 
مدیر شرکت باید بتواند کاربر را از سیستم حذف کند.
\\
\textbf{اهمیت:} \lr{Could Have}
\\
\textbf{ریسک:} متوسط
\\
\textbf{هزینه‌ی پیاده‌سازی:} متوسط

\item 
هر کاربر باید بتواند اطلاعات خودش در سیستم را ویرایش کند.
 \\
\textbf{اهمیت:} \lr{Could Have}
\\
\textbf{ریسک:} پایین
\\
\textbf{هزینه‌ی پیاده‌سازی:} پایین
\item 
مدیر شرکت باید بتواند اطلاعات هر کاربری را ویرایش کند.
 \\
\textbf{اهمیت:} \lr{Could Have}
\\
\textbf{ریسک:} پایین
\\
\textbf{هزینه‌ی پیاده‌سازی:} پایین

\item 
مدیر شرکت باید بتواند تاریخچه‌ی فعالیت‌های کاربر را از سیستم حذف کند.
\\
\textbf{اهمیت:} \lr{Won't Have}
\\
\textbf{ریسک:} بالا
\\
\textbf{هزینه‌ی پیاده‌سازی:} متوسط


\end{enumerate}


\subsection{زیرسیستم خدمت‌دهی}


\begin{enumerate}
	\item
مشتری باید امکان ثبت درخواست خدمت را داشته باشد.
	\\
	\textbf{اهمیت:} \lr{Must Have}
	\\
	\textbf{ریسک:} بالا
	\\
	\textbf{هزینه‌ی پیاده‌سازی:} بالا
	
	\item
	مشتری باید امکان حذف درخواست خدمت را داشته باشد.
	\\
	\textbf{اهمیت:} \lr{Must Have}
	\\
	\textbf{ریسک:} متوسط
	\\
	\textbf{هزینه‌ی پیاده‌سازی:} پایین
	
	\item
	متخصص باید از ثبت درخواست جدید در حوزه‌ی تخصصش از طریق دریافت پیام مطلع شود.
	\\
	\textbf{اهمیت:} \lr{Must Have}
	\\
	\textbf{ریسک:} متوسط
	\\
	\textbf{هزینه‌ی پیاده‌سازی:} بالا
	
		\item
متخصص باید امکان مشخص کردن تخصص‌های خود را داشته باشد.
	\\
	\textbf{اهمیت:} \lr{Must Have}
	\\
	\textbf{ریسک:} پایین
	\\
	\textbf{هزینه‌ی پیاده‌سازی:} پایین
	
			\item
	مشتری باید بتواند متخصصین یک حوزه را مشاهده و جست‌و‌جو کند.
	\\
	\textbf{اهمیت:} \lr{Must Have}
	\\
	\textbf{ریسک:} پایین
	\\
	\textbf{هزینه‌ی پیاده‌سازی:} پایین
	
	
	\item
	متخصص باید امکان پذیرش یک درخواست را داشته باشد.
	\\
	\textbf{اهمیت:} \lr{Must Have}
	\\
	\textbf{ریسک:} پایین
	\\
	\textbf{هزینه‌ی پیاده‌سازی:} پایین
	
	\item
	متخصص باید امکان رد یک درخواست را داشته باشد.
	\\
	\textbf{اهمیت:} \lr{Must Have}
	\\
	\textbf{ریسک:} پایین
	\\
	\textbf{هزینه‌ی پیاده‌سازی:} پایین
	
	\item
	کاربر باید امکان رد یا تایید متخصصی که درخواستش را قبول کرده داشته باشد.
	\\
	\textbf{اهمیت:} \lr{Must Have}
	\\
	\textbf{ریسک:} پایین
	\\
	\textbf{هزینه‌ی پیاده‌سازی:} پایین
	
		\item
	متخصص باید بتواند زمان آغاز و پایان ارائه خدمت را مشخص کند.
	\\
	\textbf{اهمیت:} \lr{Should Have}
	\\
	\textbf{ریسک:} متوسط
	\\
	\textbf{هزینه‌ی پیاده‌سازی:} پایین
	
	
			\item
مدیر شرکت باید بتواند تخصص‌‌های جدید به خدمات ارائه شده توسط سیستم اضافه کند.
	\\
	\textbf{اهمیت:} \lr{Should Have}
	\\
	\textbf{ریسک:} پایین
	\\
	\textbf{هزینه‌ی پیاده‌سازی:} پایین
	
				\item
مشتری باید بتواند خدمات ارائه شده را به طور دسته‌بندی شده ببیند.
	\\
	\textbf{اهمیت:} \lr{Should Have}
	\\
	\textbf{ریسک:} پایین
	\\
	\textbf{هزینه‌ی پیاده‌سازی:} پایین
	
	\item
	متخصص باید امکان پذیرش تعداد مشخصی درخواست بر حسب امتیازهایش داشته باشد.
	\\
	\textbf{اهمیت:} \lr{Should Have}
	\\
	\textbf{ریسک:} پایین
	\\
	\textbf{هزینه‌ی پیاده‌سازی:} پایین
	
		\item
	متخصص باید درآمد تخمینی از قبول کردن خدمت را مشاهده کند.
	\\
	\textbf{اهمیت:} \lr{Could Have}
	\\
	\textbf{ریسک:} بالا
	\\
	\textbf{هزینه‌ی پیاده‌سازی:} بالا
	
		\item
	مشتری امکان ثبت درخواست برای زمانی در آینده را داشته باشد.
	\\
	\textbf{اهمیت:} \lr{Could Have}
	\\
	\textbf{ریسک:} متوسط
	\\
	\textbf{هزینه‌ی پیاده‌سازی:} پایین
			\item
	مشتری امکان ثبت لوکیشن محل را روی نقشه داشته باشد.
	\\
	\textbf{اهمیت:} \lr{Could Have}
	\\
	\textbf{ریسک:} متوسط
	\\
	\textbf{هزینه‌ی پیاده‌سازی:} بالا
	
		\item
متخصص امکان تعیین زیرحوزه‌ی تخصص خود را داشته باشد.
	\\
	\textbf{اهمیت:} \lr{Could Have}
	\\
	\textbf{ریسک:} پایین
	\\
	\textbf{هزینه‌ی پیاده‌سازی:} پایین
	\item
متخصص باید امکان فیلتر کردن و مرتب‌سازی درخواست‌ها بر اساس معیارهای مختلف درخواست‌ها را داشته باشد.
	\\
	\textbf{اهمیت:} \lr{Could Have}
	\\
	\textbf{ریسک:} پایین
	\\
	\textbf{هزینه‌ی پیاده‌سازی:} متوسط
	
\end{enumerate}


\subsection{زیرسیستم ارزیابی}
\begin{enumerate}
	\item
مشتری باید امکان ثبت امتیاز در معیارهای مختلف برای خدمت دریافتی را داشته باشد.
\\
\textbf{اهمیت:} \lr{Must Have}
\\
\textbf{ریسک:} پایین
\\
\textbf{هزینه‌ی پیاده‌سازی:} پایین
	\item
مشتری باید امکان ثبت نظر در معیارهای مختلف برای خدمت دریافتی را داشته باشد.
\\
\textbf{اهمیت:} \lr{Must Have}
\\
\textbf{ریسک:} پایین
\\
\textbf{هزینه‌ی پیاده‌سازی:} پایین
	\item
متخصص باید امکان ثبت امتیاز در معیارهای مختلف برای دریافت خدمت ارائه‌شده را داشته باشد.
\\
\textbf{اهمیت:} \lr{Must Have}
\\
\textbf{ریسک:} پایین
\\
\textbf{هزینه‌ی پیاده‌سازی:} پایین
	\item
متخصص باید امکان ثبت نظر در معیارهای مختلف برای دریافت خدمت ارائه‌شده را داشته باشد.
\\
\textbf{اهمیت:} \lr{Must Have}
\\
\textbf{ریسک:} پایین
\\
\textbf{هزینه‌ی پیاده‌سازی:} پایین
	\item
مدیر شرکت باید امکان مشاهده‌ی معیارهای موجود در سیستم را داشته باشد.
\\
\textbf{اهمیت:} \lr{Must Have}
\\
\textbf{ریسک:} پایین
\\
\textbf{هزینه‌ی پیاده‌سازی:} پایین
	\item
مدیر شرکت باید امکان ویرایش و حذف معیارهای سیستم را داشته باشد.
\\
\textbf{اهمیت:} \lr{Must Have}
\\
\textbf{ریسک:} پایین
\\
\textbf{هزینه‌ی پیاده‌سازی:} متوسط

	\item
مدیر شرکت باید امکان اضافه کردن معیار جدید به سیستم را داشته باشد.
\\
\textbf{اهمیت:} \lr{Must Have}
\\
\textbf{ریسک:} پایین
\\
\textbf{هزینه‌ی پیاده‌سازی:} متوسط


	\item
کاربران عادی باید امکان انتقال مشکلات پیش آمده هنگام ثبت درخواست خدمت یا قبول خدمت در سامانه را داشته باشند.
\\
\textbf{اهمیت:} \lr{Should Have}
\\
\textbf{ریسک:} پایین
\\
\textbf{هزینه‌ی پیاده‌سازی:} پایین

\end{enumerate}

\subsection{زیرسیستم گزارش‌گیری}

\begin{enumerate}

\item
مدیر شرکت باید بتواند لیست مشتریانی که از خدمات ناراضی بودند را به همراه دلیل نارضایتی آن‌ها دریافت کند.
\\
\textbf{اهمیت:} \lr{Must Have}
\\
\textbf{ریسک:} متوسط
\\
\textbf{هزینه‌ی پیاده‌سازی:} متوسط

\item
مدیر شرکت باید بتواند لیست خدماتی که در یک بازه‌ی زمانی با کیفیت بالا یا کیفیت پایین ارائه شدند را دریافت کند.
\\
\textbf{اهمیت:} \lr{Must Have}
\\
\textbf{ریسک:} متوسط
\\
\textbf{هزینه‌ی پیاده‌سازی:} متوسط

\item
مدیر شرکت باید بتواند خدمات پرتقاضا و کم‌تقاضا در یک بازه‌ی زمانی را دریافت کند.
\\
\textbf{اهمیت:} \lr{Must Have}
\\
\textbf{ریسک:} متوسط
\\
\textbf{هزینه‌ی پیاده‌سازی:} متوسط

\item
مدیر شرکت باید بتواند لیست مرتب‌شده‌ی متخصصین بر حسب امتیاز کسب شده در یک بازه‌ی زمانی را مشاهده کند.
\\
\textbf{اهمیت:} \lr{Must Have}
\\
\textbf{ریسک:} متوسط
\\
\textbf{هزینه‌ی پیاده‌سازی:} بالا

\item
مدیر شرکت باید بتواند لیست خدمت‌های درخواست شده که در نهایت توسط هیچ متخصصی ارائه نشده‌اند و دلیل آن‌ها را دریافت کند.
\\
\textbf{اهمیت:} \lr{Must Have}
\\
\textbf{ریسک:} متوسط
\\
\textbf{هزینه‌ی پیاده‌سازی:} بالا

\item
مدیر شرکت باید بتواند لیست مرتب‌شده‌ی متخصصین بر حسب امتیاز کسب شده در یک بازه‌ی زمانی را مشاهده کند.
\\
\textbf{اهمیت:} \lr{Must Have}
\\
\textbf{ریسک:} متوسط
\\
\textbf{هزینه‌ی پیاده‌سازی:} بالا

\item
مدیر شرکت باید امکان مشاهده‌ی تاریخچه‌ی خدمات ارائه شده را داشته باشد.
\\
\textbf{اهمیت:} \lr{Should Have}
\\
\textbf{ریسک:} متوسط
\\
\textbf{هزینه‌ی پیاده‌سازی:} بالا


\item
مدیر شرکت باید امکان فیلتر و جست‌وجو در تاریخچه‌ی خدمات ارائه شده را داشته باشد.
\\
\textbf{اهمیت:} \lr{Should Have}
\\
\textbf{ریسک:} متوسط
\\
\textbf{هزینه‌ی پیاده‌سازی:} بالا

\item
مدیر فنی شرکت باید بتواند لیست مشکلات فنی گزارش‌شده‌ی سیستم را دریافت کند.
\\
\textbf{اهمیت:} \lr{Should Have}
\\
\textbf{ریسک:} متوسط
\\
\textbf{هزینه‌ی پیاده‌سازی:} متوسط



\item
مدیر فنی شرکت باید بتواند لاگ‌های سیستم در یک بازه‌ی زمانی خاص را بر حسب میزان حساسیت مشاهده کند.
\\
\textbf{اهمیت:} \lr{Could Have}
\\
\textbf{ریسک:} متوسط
\\
\textbf{هزینه‌ی پیاده‌سازی:} متوسط

\end{enumerate}

\section{نیاز‌مندی‌های غیروظیفه‌ای}

نیازمندی‌های غیروظیفه‌ای، بخش مهمی از نیازمندی‌ها هستند که تعاریف گوناگونی برای آن‌ها ذکر شده است. یکی از این تعاریف به این صورت است که «نیازمندی‌های غیروظیفه‌ای، نیازمندی‌هایی هستند که تشکیل دهنده توجیه‌های تصمیمات گرفته شده در حین طراحی بوده و محدودیت‌هایی را بر نحوه به تحقق رسیدن نیازمندی‌های وظیفه‌ای اعمال می‌کنند.

دسته‌بندی‌های زیادی برای این دسته از نیازمندی‌ها وجود دارد ولی در این جا، ما مطابق استاندارد \lr{ISO25010} تعدادی از نیازمندی‌های غیروظیفه‌ای که در این پروژه به طور خاص موضوعیت دارند را ذکر می‌کنیم:

\begin{itemize}
	\item 
	بهره‌وری عملکرد (\lr{Performance Efficiency})
	
	\begin{itemize}
		\item
		ظرفیت (\lr{Capacity}): ظرفیت به معنی حدی از پارامترهای سیستم است که در آن می‌توانیم نیازمندی‌های کاربر را برطرف کنیم. در مورد محصول ما، برای انتهای پروژه (انتهای درس)، ما ظرفیت ۵ درخواست در ثانیه را در نظر گرفته‌ایم که برای نسخه ابتدایی محصول و با محدودیت‌های سخت‌افزاری که داریم، عدد معقولی است.
		
		\item 
		رفتار زمانی (\lr{Time Behaviour} ): رفتار زمانی به معنی حد مناسب برای مواردی نظیر نرخ گذردهی، سرعت پاسخ دادن و سرعت پردازش است. برای محصول فعلی، ما نرخ پاسخ‌دهی ۱ ثانیه به ازای ریکوئست‌های عادی که شامل جست‌وجو‌های سنگین روی دیتابیس نباشند و ۵ ثانیه برای کوئری‌هایی که جست‌وجوی سنگینی روی دیتابیس داشته باشند را در نظر گرفته‌ایم.
		
	\end{itemize}

	\item
	سازگاری (\lr{Compatibility})
	\begin{itemize}
		\item
		قابلیت همکاری (\lr{Interoperability}): این خصیصه به معنی این است که دو یا چند سیستم، محصول یا زیر‌سیستم بتوانند به خوبی اطلاعات را با یکدیگر رد و بدل کرده و از آن استفاده کنند. در مورد محصول ما، زیرسیسیتم‌های مختلف آن نظیر گزارش‌گیری، سرویس‌دهی و کاربری، به اطلاعات یکدیگر نیاز دارند و باید سیستم طوری طراحی شود که با کمترین سربار، از اطلاعات یک زیرسیستم بتوانیم در زیرسیستم‌های دیگر استفاده نماییم.
		
	\end{itemize}

\item
قابلیت استفاده (\lr{Usability})

\begin{itemize}

\item 
قابلیت یادگیری (\lr{Learnability}): این خصیصه بیانگر این است که کاربر چقدر راحت می‌تواند کار با سیستم را به شکل بهینه، کارا، بدون نگرانی و با رضایت یاد بگیرد. برای تحقق این مورد، باید رابط کاربری و تجربه کاربری سیستم به شکلی مناسب و تا حد امکان ساده طراحی بشود. همچنین از طریق مستندات آموزشی، نحوه کارکرد سیستم در مواردی که ممکن است ابهام داشته باشند، به کاربر آموزش داده خواهد شد.

\item
محافظت در مقابل خطاهای کاربر (\lr{User Error Protection}): این مورد به معنی این است که محصول باید طوری طراحی شود که از کاربر در مقابل خطاهایی که ممکن است خود کاربر مرتکب شود محافظت کند. این کار با بررسی دقیق تک تک پروسه‌های سیستم و تشخیص تمامی روندهای ممکن برای هر بخش و پیش‌بینی پیام‌های مناسب برای آن‌ها قابل دستیابی است.

\item 
دسترس‌پذیری (\lr{Accessibility}): دسترس‌‌پذیری به معنی این است که کاربران با سطح توانایی‌های مختلف (مثلا افرادی که دچار مشکلات بینایی هستند و...) بتوانند به خوبی از سیستم استفاده کنند. برای این سیستم،‌ قصد داریم تا حد معقولی از دسترس‌پذیری را در اختیار کاربران قرار بدهیم تا افراد بیش‌تری امکان استفاده از محصول ما را داشته باشند. 


\item 
زیبایی رابط کاربری (\lr{User Interface Aesthetics}): رابط کاربری باید به گونه‌ای طراحی شود که عموم کاربران بتوانند به خوبی با آن تعامل کرده و از کار با آن لذت ببرند. 


\end{itemize}


\item
قابلیت اطمینان (\lr{Reliability})

\begin{itemize}
	\item 
	
	در دسترس بودن (\lr{Availability}): به معنی در دسترس و عملیاتی بودن سیستم در زمان‌های لازم است. برای این موضوع سعی داریم تا حد امکان سیستم بدون مشکلات جدی پیاده‌سازی بشود تا به جز زمان‌های برنامه‌ریزی شده، بتوانیم بعد از استقرار آن Uptime $90$ درصد داشته باشیم.
	
	\item
	قابلیت بازیابی (\lr{Recoverability}): این موضوع به معنی این است که بعد از ایجاد یک خطا در سیستم، بتوانیم به سادگی به وضعیت پایداری در سیستم بازگردیم. برای این موضوع سیستم‌های بک‌آپ‌گیری خودکار را استفاده خواهیم کرد تا در صورت وقوع مشکل بتوانیم به راحتی به وضعیت پایدار قبلی بازگردیم.
\end{itemize}

\item
امنیت (\lr{Security})

\begin{itemize}
	\item 
	محرمانگی (\lr{Confidentiality}): این یعنی به هر داده، تنها افرادی که مجوز آن را دارند دسترسی داشته باشند. این کار از طریق تعریف سطوح دسترسی مختلف در سیستم امکان پذیر خواهد بود.
	
	\item 
	درستی (\lr{Integrity}): درستی به معنی این است که افرادی که دسترسی لازم را ندارند، نتوانند تغییراتی در داده‌ها اعمال کنند. این موضوع نیز از طریق تعریف درست سطوح دسترسی برای دسته‌های مختلف کاربران امکان پذیر خواهد بود.
	\item 
	مسئولیت‌پذیری (\lr{Accountability}): مسئولیت پذیری بدین معنی است که کارهایی که توسط یک موجودیت انجام شده است را بتوانیم به طور یکتا اثبات کنیم که توسط آن موجودیت انجام شده. برای دستیابی به این مورد می‌توانیم \lr{Log} تک تک کارهایی که یک کاربر یا سایر موجودیت‌های سیستم انجام می‌دهند را نگه‌داریم.
\end{itemize}

\item 
نگهداشت‌پذیری (\lr{Maintainability})
\begin{itemize}
	
	\item
	ماژولار بودن (\lr{Modularity}): ماژولار بودن به این معنیست که سیستم از اجزای جداگانه‌ای تشکیل شده باشد که تغییر در هر کدام نیازمند تغییرات کمی در سایر ماژول‌ها باشند. این موضوع نیازمند این است که افزایش \lr{Cohesion} و کاهش \lr{Coupling} در هنگام طراحی و پیاده‌سازی سیستم مورد توجه قرار بگیرد.
	
	\item 
	قابلیت آنالیز (\lr{Analysability}): این قابلیت بدین معنی است که اجزای سیستم طوری شفاف در کنار هم قرار گرفته باشند که بتوان به طرز کارا، آن‌ها را آنالیز کرده و وظایف هر یک را تشخیص داد و در مواقع بروز خطا، به راحتی منشا آن را پیدا کرد.
	
	\item 
	قابلیت آزمون (\lr{Testability}): اجزای سیستم باید به شکلی طراحی و پیاده‌سازی بشوند که بتوان عملکرد آن‌ها را با سنجه‌های مختلف در حوزه‌های گوناگون سنجید.
	
\end{itemize}


\item
قابلیت جابه‌جایی (\lr{Portability})

\begin{itemize}
	\item
	قابلیت نصب (\lr{Installability}):
	این موضوع به معنی این است که قابلیت نصب یا استقرار سیستم به شکل کارا و راحت وجود داشته باشد. در مورد این پروژه نیز باید این موضوع مد نظر قرار بگیرد تا با استفاده از ابزارهای موجود، استقرار و نصب سیستم هم برای سرور و هم استفاده از رابط کاربری آن برای کاربران عادی بدون دردسر و به شکلی راحت امکان پذیر باشد.
	
\end{itemize}
\end{itemize}

