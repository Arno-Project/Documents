
\chapter{موارد کاربرد}


\section{تعریف کنش‌گر‌ها}


\section{نمودار موارد کاربرد}


\newpage
\section{فهرست موارد کاربرد}


\subsection{زیرسیستم کاربری}

\usecase
{ثبت نام}
{۱}
{مشتری یا متخصص، در سایت ثبت نام می‌کنند تا بتوانند از امکانات سایت استفاده کنند.}
{مشتری، متخصص}
{}{کاربر (مشتری یا متخصص) وارد حساب کاربری خود نشده باشد (لاگین نکرده باشد).}
{
\vspace*{-0.6cm}
\begin{enumerate}
	\item 
	کاربر اقدام به ثبت نام در سایت به عنوان متخصص یا مشتری می‌کند.
	\item
	\textbf{تا زمانی که} اطلاعات کاربر به طور کامل وارد نشده است:
	
	\begin{enumerate}[label=\theenumi.\arabic*.]
	\item
	کاربر اطلاعات هویتی شامل نام و نام‌خانوادگی و همچنین شماره تلفن همراه و ایمیل را وارد می‌کند.
	\item 
	کاربر رمز عبور خود را وارد می‌کند.
	
	\item 
	\textbf{اگر} کاربر قصد ثبت نام به عنوان متخصص را داشت:
	\begin{enumerate}
		\item 
		کاربر تخصص‌هایی که در آن‌ها مهارت دارد را وارد می‌کند.
	\end{enumerate}

	\item 
	کاربر اطلاعات خود را ثبت می‌کند.
	
	\item 
	سیستم صحت کلی اطلاعات کاربر را کنترل می‌کند.
	\end{enumerate}
	
	\item 
	سیستم ثبت‌نام موفق را به اطلاع کاربر می‌رساند.
	
	
\end{enumerate}
}
{یک اکانت برای کاربر ساخته می‌شود. در صورتی که کاربر به عنوان متخصص ثبت‌نام کرده باشد، در وضعیت در انتظار تایید توسط مدیر قرار می‌گیرد.}
{\begin{itemize}
		\vspace*{-0.6cm}
		\item انصراف: در صورت انصراف، ثبت نام کاربر لغو می‌شود
		\item معتبر نبودن ایمیل در مرحله ۲.۱.: در این صورت سیستم در قالب پیامی کاربر را مطلع می‌کند.
		
		
\end{itemize}}
{مورد کاربرد: ثبت نام}


\usecase
{ورود}
{۲}
{کاربرانی که در سایت ثبت‌ نام کرده‌اند یا از ابتدا اکانت دارند (شامل مدیر، مشتری و متخصص) وارد حساب کاربری خود می‌شوند.}
{همه کاربران (مدیر، مشتری و متخصص)}
{}
{
		\begin{itemize}
		\item
		کاربر در سیستم لاگین نکرده باشد.
		\item
		اکانتی برای کاربر در سایت وجود داشته باشد.
	\end{itemize}
	
}
{
\begin{enumerate}
	\item 
	کاربر درخواست وارد شدن به حساب کاربری خود را می‌کند.
	
	
	\item 
	کاربر ایمیل و پسورد خود را وارد می‌کند.
	
	\item
	سیستم درستی اطلاعات وارد شده را بررسی می‌کند.
	
	\item
	کاربر احراز هویت شده و وارد حساب کاربری خود می‌شود.
\end{enumerate}
}{اگر اطلاعات وارد شده درست باشد، کاربر وارد حساب کاربی خود می‌شود.}
{	
	
	\begin{itemize}
	\item
	 انصراف: در صورت انصراف، ورود کاربر لغو می‌شود.
	
	\item
	اشتباه بودن اطلاعات وارد شده: اگر در مرحله ۳ سیستم متوجه اشتباه بودن اطلاعات کاربر شود، با صادر کردن اخطار این موضوع را به کاربر اطلاع می‌دهد. کاربر به مرحله ۲ باز می‌گردد.
	\end{itemize}
}
{مورد کاربرد: ورود }



\usecase
{جست‌وجوی کاربران}
{۳}
{مدیر شرکت با وارد کردن فیلترهای مختلف به جست‌و‌جوی کاربران می‌پردازد}
{مدیر شرکت}
{}
{
	\begin{itemize}
	\item
	مدیر در سیستم لاگین کرده باشد.
	
	\end{itemize}
 }
{
\begin{enumerate}
	\item 
	مدیر وارد قسمت جست‌وجوی کاربران می‌شود.
	
	\item 
	مدیر اطلاعات لازم برای فیلتر‌های جست‌وجو - شامل نام،‌ نام‌خانوادگی، نوع کاربر (متخصص یا مشتری)، تخصص‌ها در صورت انتخاب نوع متخصص، ایمیل و شماره تلفن- را وارد می‌کند.
	
	\item
	سیستم براساس فیلتر‌های وارد شده در لیست کاربران جست‌وجو می‌کند.
	
	\item 
	کاربرانی که با فیلتر‌های وارد شده تطابق داشته باشند، به عنوان خروجی داده می‌شوند.
	
	\item
	شامل (\lr{Include}) مورد کاربری «مشاهده کاربران»
\end{enumerate}
}
{}
{
\begin{itemize}
	\item	
	انصراف: مدیر از حالت جست‌وجوی کاربران خارج می شود.
	
	\item 
	پیدا نشدن هیچ کاربری در مرحله ۳: خطایی در مورد یافت نشدن کاربری با مشخصات وارد شده نمایش داده می‌شود.
\end{itemize}
}
{مورد کاربرد: جست‌وجوی کاربران}



\usecase
{مشاهده کاربران}
{۴}
{مدیر شرکت اطلاعات کاربران مشخص شده را مشاهده می‌کند.}
{مدیر شرکت}
{}
{
		\begin{itemize}
		\item
		مدیر در سیستم لاگین کرده باشد.
		
		\item
		‌کاربرانی که قرار است نمایش داده شوند مشخص شده باشند.
	\end{itemize}
}
{
\begin{enumerate}
	\item 
	سیستم لیست (آی‌دی) کاربرانی که قرار است نمایش داده شوند را دریافت می‌کند.
	
	\item
	سیستم کاربران مشخص شده را به همراه تمامی صفات آنان به مدیر نمایش می‌دهد.
\end{enumerate}
}
{
}
{
	\begin{itemize}
		\item 
		وجود نداشتن برخی از آی‌دی‌های مشخص شده: سیستم اطلاعات کاربری که وجود ندارد را نمایش نمی‌دهد و آن را نادیده می‌گیرد.
	\end{itemize}
}
{مورد کاربرد: مشاهده کاربران}


\usecase
{تایید متخصص}
{۵}
{مدیر شرکت متخصصی که ثبت نام کرده است را تایید یا رد می‌کند}
{مدیر شرکت}
{}
{
	\begin{itemize}
	\item 
	کاربر متخصص در سیستم ثبت‌نام کرده باشد و هنوز تایید نشده باشد.
	
	\item
مدیر در سیستم لاگین کرده باشد.
\end{itemize}
}
{
\begin{enumerate}
	\item 
	شامل (\lr{Include}) مورد کاربری «جست‌وجوی کاربران»
	\item
	سیستم لیست متخصصانی که تایید نشده‌اند را نمایش می‌دهد.
	
	\item 
	مدیر شرکت متخصصی که قصد تایید یا رد آن را دارد انتخاب می‌کند.
	
	\item 
	مدیر شرکت بسته به اطلاعات متخصص، رد یا تایید متخصص را مشخص می‌کند.
\end{enumerate}
}
{\begin{itemize}
	\item
	وضعیت متخصص بسته به انتخاب مدیر، به حالت رد شده یا تایید شده در می‌آید.
\end{itemize}}
{
\begin{itemize}
	\item انصراف: مدیر از قسمت تایید متخصص خارج می‌شود.
\end{itemize}
}
{مورد کاربرد: تایید متخصص}



\usecase
{اضافه کردن مدیر جدید}
{7}
{یک مدیر شرکت می‌تواند حساب کاربری برای مدیر جدیدی در شرکت ایجاد کند.}
{مدیر شرکت}
{}
{
	\begin{itemize}
		\item
		مدیر شرکت در سیستم لاگین کرده باشد.
		
	\end{itemize}
}
{
\begin{enumerate}
	\item 
	مدیر شرکت وارد قسمت اضافه کردن مدیر جدید می‌شود.
	
\item
\textbf{تا زمانی که} اطلاعات کاربر به طور کامل وارد نشده است:

\begin{enumerate}[label=\theenumi.\arabic*.]
	\item
	کاربر اطلاعات هویتی شامل نام و نام‌خانوادگی و همچنین شماره تلفن همراه و ایمیل و پسورد را وارد می‌کند.

	
	\item 
	سیستم صحت کلی اطلاعات مدیر شرکت جدید را کنترل می‌کند.
\end{enumerate}

	\item 
سیستم ثبت‌نام موفق را به اطلاع مدیر شرکتی که مراحل اضافه کردن را انجام داده می‌رساند.

\end{enumerate}
}
{
حساب کاربری برای مدیر شرکت جدید ایجاد می‌شود.
}
{\begin{itemize}
		\vspace*{-0.6cm}
		\item انصراف: در صورت انصراف، ایجاد مدیر جدید لغو می‌شود
		\item معتبر نبودن ایمیل در مرحله ۲.۱.: در این صورت سیستم در قالب پیامی مدیر شرکتی که در حال اضافه کردن مدیر شرکت جدید بوده را مطلع می‌کند.
		
		
\end{itemize}}
{مورد کاربرد: اضافه کردن مدیر جدید}

\usecase
{فعال یا غیرفعال کردن متخصص}
{۶}
{مدیر شرکت، یک متخصص را فعال یا غیرفعال می‌کند تا امکان یا عدم امکان خدمات رسانی توسط آن متخصص را مشخص کند.}
{مدیر شرکت}
{}
{
\begin{itemize}
	\item 
	مدیر شرکت در سیستم لاگین کرده باشد.
	
	\item
	متخصص مد نظر در وضعیت تایید شده باشد.
\end{itemize}
}
{
\begin{enumerate}
	\item 
	شامل (\lr{Include}) مورد کاربری «جست‌وجوی کاربران»
	\item
	سیستم لیست متخصصانی که تایید نشده‌اند را نمایش می‌دهد.
	
	\item
	\textbf{اگر} متخصص در وضعیت فعال باشد:
	\begin{enumerate}
		\item 
		مدیر شرکت وضعیت آن را به حالت غیرفعال تغییر می‌دهد.
	\end{enumerate}

\item
	\textbf{در غیر این صورت } اگر متخصص در وضعیت غیر فعال باشد:
\begin{enumerate}[label=\theenumi.\arabic*.]
	\item 
	مدیر شرکت وضعیت آن را به حالت فعال تغییر می‌دهد.
\end{enumerate}
\end{enumerate}
}
{\begin{itemize}
		\item وضعیت فعال یا غیرفعال بودن متخصص تغییر می‌کند. (در وضعیت غیرفعال متخصص امکان قبول درخواست را ندارد.)
\end{itemize}}
{
\begin{itemize}
	\item انصراف: در صورت انصراف، تغییر وضعیت متخصص لغو می‌شود
\end{itemize}
}
{مورد کاربرد: فعال یا غیرفعال کردن متخصص}



\subsection{زیرسیستم سرویس‌دهی}

\usecase
% name
{}
% id
{}
% brief description
{}
% primary actos
{}
% secondary actors
{}
% preconditons
{}
% main flow
{
	\vspace*{-0.6cm}
	\begin{enumerate}
		\item 
	\end{enumerate}
}
% postconditions
{}
% alternative flows
{
	\begin{itemize}
		\vspace*{-0.6cm}
		\item 
\end{itemize}
}
% table name
{
	مورد کاربرد:
}


\subsection{زیرسیستم بازخورد}

\subsection{زیرسیستم گزارش‌گیری}
