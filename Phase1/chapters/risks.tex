
\chapter{ریسک‌ها}

ریسک و مخاطرات، جزئی جدایی ناپذیر و ذاتی از فعالیت‌های تولید نرم‌افزار هستند و از سویی ریسک‌پذیری برای برای پیشرفت ضروری است. تیم تولید یک نرم‌افزار باید بداند که ریسک نه بد است و نه خوب، اما همیشه در پروژه وجود دارد و برخورد مناسب با آن باعث می‌شود که احتمال موفقیت پروژه افزایش یابد.


تمامی پارادایم‌های مدیریت ریسک، شامل فعالیت‌های زیر هستند:

\begin{itemize}
	\item
	 \textbf{شناسایی}: پیش از مدیریت ریسک‌ها، باید آن‌ها را شناسایی کرد. شناسایی باعث می‌شود که ریسک‌ها قبل از آن‌ که تبدیل به مشکل بشوند، آشکار بشوند. 
	 
	 \item
	 \textbf{تحلیل}: تحلیل به معنی تبدیل داده‌هایی که از ریسک داریم به اطلاعاتی که در تصمیم‌گیری می‌تواند مفید واقع شود است. مرحله تحلیل پایه اصلی کار بر روی ریسک «درست» را برای مدیر پروژه فراهم می‌کند.
	 
	 \item 
	 \textbf{برنامه‌ریزی}: در مرحله برنامه‌ریزی، اطلاعتی که از ریسک وجود دارد تبدیل به تصمیم و عمل می‌شود. بسته به نوع ریسک،  برنامه‌های مختلفی نظیر کاهش ریسک، پرهیز از ریسک، پذیرش ریسک، مطالعه بیش‌تر ریسک و... می‌توان ترتیب داد.
	 
	 \item
	 پیگیری کردن: پیگیری کردن به معنی مانیتورینگ و زیرنظر گرفتن دائمی وضعیت ریسک‌ها و اعمالی که برای مقابله با آنان پیش‌بینی شده،‌ است.
	 
	 \item 
	 کنترل: کنترل ریسک به معنی اصلاح انحراف‌هایی است که از برنامه تنظیم شده برای مقابله با ریسک ایجاد شده‌اند.
	 
	 \item
	 ارتباطات: ارتباط برقرار کردن با اعضای تیم در مورد ریسک، در هسته مدل‌های مدیریت ریسک قرار دارد. بدون ارتباطات موثر هیچ پاردایم مدیریت ریسکی قابل اجرا نخواهد بود.
	 
\end{itemize}


با توجه به موارد گفته شده در بالا، در ادامه این مستند به شناسایی ریسک‌هایی که ممکن است پروژه را تحت تاثیر قرار دهد و همچنین در مواردی که ممکن بوده، بعضی از راه‌حل‌های احتمالی آن می‌پردازیم. پیش از بررسی ریسک‌ها باید بدانیم که ریسک‌های تولید نرم‌افزار را می‌توان در سه دسته کلی زیر تقسیم‌بندی کرد:

\begin{itemize}
	\item
	 مهندسی محصول: شامل ابعاد فنی کاری که قرار است انجام بشود.
	
	\item
	 محیط توسعه: شامل متودها، فرآیندها و ابزارهایی برای تولید محصول استفاده می‌شوند.
	
	\item
	 محدودیت‌های برنامه: شامل محدودیت‌های سازمانی، قراردادی و عملیاتی که معمولا خارج از کنترل مدیران محلی سیستم است.
\end{itemize}

\section{فهرست ریسک‌ها}

\subsection{ریسک‌های حوزه مهندسی محصول}

ریسک‌های حوزه مهندسی محصول را می‌توان در ۶ دسته کلی تقسیم‌بندی کرد. در ادامه به بررسی ریسک‌هایی از هر کدام از دسته‌ها می‌پردازیم.


\subsubsection{نیازمندی‌ها}

\begin{itemize}
	
	\item
	\textbf{پایدار نبودن و امکان تغییر نیازمندی‌ها}
	
	
	توضیح \hspace*{1cm}  پایداری نیازمندی‌ها، به این درجه تغییرات آن‌ها و تاثیراتی است که تغییرات آن‌ها بر کیفیت، برنامه‌زمانی، طراحی، تولید و تست برنامه وارد می‌کند. با توجه به مختصر بودن پروپوزال پروژه و همچنین عدم شناخت جامع تیم بر حوزه پروژه،‌ امکان تغییر در نیازمندی‌ها وجود دارد.
	
	
	راه‌حل \hspace*{1cm}  راه‌حلی که برای این موضوع می‌توان در نظر گرفت، در ابتدا افزایش دانش تیم بر روی موضوع کلی پروژه است تا به مرور پایداری نیاز‌مندی‌ها افزایش یابد. هم با تعامل بیش‌تر با مشتری (دستیاران آموزشی)، نیازمندی‌های اساسی‌تر به طور دقیق‌تر مشخص شوند تا احتمال تغییرات ناگهانی آنان کم شود.
	
	\item 
\textbf{	کامل نبودن نیازمندی‌ها}
	
	توضیح \hspace*{1cm}  
	کامل نبودن نیازمندی‌ها،  می‌تواند به دلایلی نظیر عدم شناخت تیم، نبود وقت‌ کافی برای شناسایی دقیق نیازمندی‌ها و یا عدم تعامل مناسب با مشتری ایجاد بشود. البته طبیعتا هیچ‌ وقت نیازمندی‌ها در همان ابتدای کار به طور کامل مشخص نمی‌شوند ولی باید تلاش کرد که کمتر از حد مورد انتظار هم نباشند.
	
	راه‌حل \hspace*{1cm}  باید سعی کرد با افزایش شناخت بر روی حوزه کاری و همچنین افزایش تعاملات با مشتری، نیازمندی‌ها را به مرور کامل‌تر کرد.
	
	
	
	\item 
	\textbf{	واضح نبودن نیازمندی‌ها}
	
	
	توضیح \hspace*{1cm}  
واضح نبودن نیازمند‌ها و تعریف مبهم یا نادقیق ‌آن‌ها می‌تواند تیم را در فاز توسعه دچار مشکل کند. از دلاین وقوع این مشکل می‌توان به نبود ارتباطات مناسب درون تیم تولید و طراحی محصول، نبود ارتباط بین این تیم با مشتری‌ها و نبود شناخت کافی نسبت به حوزه کاری اشاره کرد.
	
	راه‌حل \hspace*{1cm}  باید تعاملات درون تیمی برای مشخص‌تر کرن نیازمندی‌ها و همچنین تعاملات با مشتری افزایش یابد.
	
	
	\item 
	\textbf{غیرقابل پیاده‌سازی بودن بعضی از نیازمندی‌ها}
	
		توضیح \hspace*{1cm}  
	قابل پیاده‌سازی بودن، به نوعی بیانگر سختی فنی یا عملیاتی پیاده‌سازی نیازمندی‌ها است. گاهی اوقات دو نیازمندی به تنهایی قابل پیاده‌سازی هستند ولی در کنار یکدیگر، امکان پیاده‌سازی آن‌ها حتی از لحاظ تئوری هم وجود ندارد. چنین مواردی می‌تواند در مرحله پیاده‌سازی ایجاد مشکل بکند.
	
	راه‌حل \hspace*{1cm}  
	بهترین راه‌حل این است که در هنگام تهیه نیازمندی‌ها به طور دقیق‌تر امکان‌پذیر بودن آنان هم به طور مجزا و علی‌الخصوص در ترکیب با یکدیگر بررسی بشود. این موضوع تا حدی نیازمند این است که تیم به شناخت خوبی از توانایی‌های فنی خودش هم برسد.
	
	
	
	\item 
	\textbf{مسبوق به سابقه نبودن بعضی از نیازمندی‌ها}
	
		توضیح \hspace*{1cm}  
بعضی از نیازمندی‌ها ممکن است طوری باشند که تا به حال در هیچ سیستمی پیاده‌سازی نشده باشند و یا بسیار فراتر از توان فنی تیم باشند. هر چند بخش‌های مختلف پروژه به طور جداگانه به نظر نمی‌رسد دارای چنین مشکلی باشد ولی در ترکیب با هم ممکن است تیم به اشتباه چنین نیازمندی‌هایی را اضافه کند که باعث هدررفت زمان و عدم پیشرفت پروژه می‌شوند.

راه‌حل \hspace*{1cm}  
تیم باید با پیدا کردن درک کامل از حوزه مسئله و همچنین توانایی فنی خود، در هنگام تعریف نیازمندی‌ها به توان فنی خود و همچنین وجود نمونه‌های مشابهی که چنین نیازمندی‌‌ای را پیاده‌سازی کرده باشند توجه کند تا در دام نیازمندی‌های عجیب و غریبی که فراتر از توان تیم است گیر نیفتد.
	
	
	\item 
\textbf{بزرگ شدن بی‌رویه پروژه}

توضیح \hspace*{1cm}  
با توجه به ذات پیچیده پروژه، امکان این که تیم با تعریف نیازمندی‌های فراوان و جزئی گوناگون ابعاد پروژه را بیش‌ از اندازه بزرگ کند وجود دارد. این اتفاق می‌تواند باعث ایجاد چالش‌های فنی و مدیریتی در زمینه‌های مختلف نظیر تولید، زمان‌بندی، تحلیل وابستگی‌های بین اجزای سیستم و مجتمع‌سازی اجزای آن بشود.


راه‌حل \hspace*{1cm} 
برای حل این مشکل، باید در حین تعیین نیازمندی‌ها تاثیری که آن‌ها بر پروژه به عنوان ساختار کلی مورد بحث ما می‌گذارند و پیچیدگی‌هایی که ایجاد می‌کنند توجه داشت. این موضوع مطمئنا در ابتدا با توجه به شناخت‌ کمتر از پروژه کمی دشوار خواهد بود؛ ولی به تدریج با  افزایش تجربه تیم امکان انجام این بررسی‌ها به شکل موثر‌تری فراهم می‌شود.


	
\end{itemize}

\subsubsection{طراحی}


\begin{itemize}
	\item 
	\textbf{سختی طراحی ساختار مناسب برای بعضی نیازمندی‌ها}
	
	توضیح \hspace*{1cm} 
	ممکن است طراحی معماری مناسب برای برخی از نیازمندی‌ها، بسیار دشوار شود و یا سیستم‌هایی طراحی شوند که امکان پیاده‌سازی عملی آنان وجود نداشته باشد.
	
		
	راه‌حل \hspace*{1cm} 
	راه‌حلی که برای این موضوع به ذهن می‌رسد، افزایش دانش در حوزه طراحی سیستم است. همچنین در مواردی که واقعا امکان طراحی ساختار ساده برای یک نیازمندی وجود ندارد، ممکن است لازم باشد با هماهنگی مشتری تغییراتی ساده‌کننده در نیازمندی داده شود تا امکان طراحی آن میسر بشود.
	
	
	
		\item 
	\textbf{برآورده نشدن کارایی لازم}
	
	
		توضیح \hspace*{1cm} 
	کارایی (پرفرمنس)، معیاری اساسی است و مواردی نظیر سرعت پاسخ و توان عملیاتی در مراحل طراحی محصول باید لحاظ بشود. ممکن است محصول طراحی شده نتواند نیازهای کارایی مورد انتظار را برآورده سازد.
	
	راه‌حل \hspace*{1cm} 
	باید در هنگام طراحی، به طور دقیق از لحاظ کارایی سیستم تحلیل شده و همچنین در مراحل تولید، تا حد امکان از لحاظ توان فنی تیم، سیستم‌های نظارتی برای کارایی سیستم قرار داده شوند تا در صورت وجود مشکلی در طراحی که باعث کاهش کارایی در مرحله تولید شده است، قبل از گسترش این مشکل برای رفع آن چاره‌ای اندیشیده شود.
	
	
	\item
	\textbf{طراحی سیستم به شکل غیرقابل تست}
	
	
	توضیح \hspace*{1cm} 
	تست‌پذیری سیستم، از جمله مواردی است که علاوه بر این که در فاز پیاده‌سازی باید به آن توجه شود، باید از همان ابتدا و در زمان طراحی هم مورد توجه باشد. اگر سیستم به شکل بسیار در‌هم‌تنیده طراحی بشود، تست‌پذیری آن بسیار کم بشود.
	
	
	راه‌حل \hspace*{1cm} 
	راه‌حلی که فعلا به ذهن می‌رسد این است که در هنگام طراحی، تست‌پذیر بودن آن به شکل جداگانه بررسی شود تا از بابت این که طرح داده شده در هنگام پیاده‌سازی تست پذیر است مطمئن باشیم.
	
	
	
\end{itemize}

\subsubsection{کد و تست}



\begin{itemize}
	
	
	\item 
	\textbf{عدم توجه به تست واحد}
	
	
	توضیح \hspace*{1cm} 
تست واحد یکی از مهم‌ترین عواملی است که باعث حفظ کیفیت کد و افزایش قابلیت نگه‌داشت آن می‌شود. از این رو برنامه‌ریزی برای تهیه تست‌های دقیق و مناسب ضرورت دارد ولی این احتمال وجود دارد که به دلیل این که اثر آن به طور آنی احساس نمی‌شود، مورد بی‌توجهی قرار بگیرد.
	
	راه‌حل \hspace*{1cm} 
باید در هنگام پیاده‌سازی هر قسمت، به طور موازی به طراحی تست‌های مناسب برای آن فکر کرده و تا زمانی که تست‌های مناسب برای یک قسمت طراحی نشده‌اند، آن قسمت را به عنوان «تمام شده»‌ در نظر نگیریم.
	
	
	
	
\end{itemize}


\section{اولویت‌بندی ریسک‌ها}

