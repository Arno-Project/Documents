
% -------------------------------------------------------
%  Common Styles and Formattings
% -------------------------------------------------------


\usepackage{amssymb,amsmath}
\usepackage[colorlinks,linkcolor=blue,citecolor=blue]{hyperref}
\usepackage[usenames,dvipsnames]{pstricks}
\usepackage{graphicx,subfigure,wrapfig}
\usepackage{geometry,fancyhdr}
\usepackage[mathscr]{euscript}
\usepackage{multicol}
\usepackage{array}
\usepackage{enumitem}
\usepackage{setspace}
\usepackage{longtable}

\usepackage{algorithmicx,algorithm}

\usepackage[localise=on,extrafootnotefeatures]{xepersian}
\usepackage[noend]{algpseudocode}


%------------------------ Algorithm ------------------------------------

\newenvironment{الگوریتم}[1]
	{\bigskip\bigskip\begin{algorithm}\caption{#1} \label{الگوریتم: #1}\vspace{0.5em}\begin{algorithmic}[1]}
	{\end{algorithmic}\vspace{0.5em}\end{algorithm}\bigskip}
	

\renewcommand{\algorithmicfor}{{به ازای}}
\renewcommand{\algorithmicwhile}{{تا وقتی}}
\renewcommand{\algorithmicdo}{\hspace{-.2em}:}
\renewcommand{\algorithmicif}{{اگر}}
\renewcommand{\algorithmicthen}{\hspace{-.2em}:}
\renewcommand{\algorithmicelse}{{در غیر این صورت:}}
%\renewcommand{\algorithmicelsif}{{در غیر این صورت اگر: }}
\renewcommand{\algorithmicreturn}{{برگردان}}
\renewcommand{\algorithmiccomment}[1]{$\triangleleft$ \emph{#1}}
\renewcommand{\algorithmicrequire}{\textbf{ورودی:}}
\renewcommand{\algorithmicensure}{\textbf{خروجی:}}

\newcommand{\اگر}{\If}
\newcommand{\وگرنه}{\Else}
\newcommand{\وگر}{\ElsIf}
\newcommand{\پایان‌اگر}{\EndIf}
\newcommand{\به‌ازای}{\For}
\newcommand{\پایان‌به‌ازای}{\EndFor}
\newcommand{\تاوقتی}{\While}
\newcommand{\پایان‌تاوقتی}{\EndWhile}
\newcommand{\دستور}{\State}
\newcommand{\دستورک}{\Statex}
\newcommand{\توضیحات}{\Comment}
\newcommand{\برگردان}{\Return}
\renewcommand{\ورودی}{\Require}
\newcommand{\خروجی}{\Ensure}



% -------------------- Page Layout --------------------


\newgeometry{top=3.5cm,bottom=3.5cm,left=2.5cm,right=3cm,headheight=25pt}

\renewcommand{\baselinestretch}{1.4}
\linespread{1.6}
\setlength{\parskip}{0.45em}

\fancyhf{}
\rhead{\leftmark}
\lhead{\thepage}


% -------------------- Fonts --------------------

\settextfont[
Scale=1.09,
Extension=.ttf, 
Path=styles/fonts/,
BoldFont=XB NiloofarBd,
ItalicFont=XB NiloofarIt,
BoldItalicFont=XB NiloofarBdIt
]{XB Niloofar}

\setdigitfont[
Scale=1.09,
Extension=.ttf, 
Path=styles/fonts/,
BoldFont=XB NiloofarBd,
ItalicFont=XB NiloofarIt,
BoldItalicFont=XB NiloofarBdIt
]{XB Niloofar}

\defpersianfont\sayeh[
Scale=1,
Path=styles/fonts/
]{XB Kayhan Pook}


% -------------------- Styles --------------------


\SepMark{-}
\renewcommand{\labelitemi}{$\small\bullet$}



% -------------------- Environments --------------------


\newtheorem{قضیه}{قضیه‌ی}[chapter]
\newtheorem{لم}[قضیه]{لم}
\newtheorem{ادعا}[قضیه]{ادعای}
\newtheorem{مشاهده}[قضیه]{مشاهده‌ی}
\newtheorem{نتیجه}[قضیه]{نتیجه‌ی}
\newtheorem{مسئله}{مسئله‌ی}[chapter]
\newtheorem{تعریف}{تعریف}[chapter]
\newtheorem{مثال}{مثال}[chapter]


\newenvironment{اثبات}
{\begin{trivlist}\item[\hskip\labelsep{\em اثبات.}]}
	{\leavevmode\unskip\nobreak\quad\hspace*{\fill}{\ensuremath{{\square}}}\end{trivlist}}

\newenvironment{alg}[2]
{\begin{latin}\settextfont[Scale=1.0]{Times New Roman}
		\begin{algorithm}[t]\caption{#1}\label{algo:#2}\vspace{0.2em}\begin{algorithmic}[1]}
			{\end{algorithmic}\vspace{0.2em}\end{algorithm}\end{latin}}


% -------------------- Titles --------------------


\renewcommand{\listfigurename}{فهرست شکل‌ها}
\renewcommand{\listtablename}{فهرست جدول‌ها}
\renewcommand{\bibname}{\rl{{مراجع}\hfill}} 


% -------------------- Commands --------------------


\newcommand{\IN}{\ensuremath{\mathbb{N}}} 
\newcommand{\IZ}{\ensuremath{\mathbb{Z}}} 
\newcommand{\IQ}{\ensuremath{\mathbb{Q}}} 
\newcommand{\IR}{\ensuremath{\mathbb{R}}} 
\newcommand{\IC}{\ensuremath{\mathbb{C}}} 

\newcommand{\set}[1]{\left\{ #1 \right\}}
\newcommand{\seq}[1]{\left< #1 \right>}
\newcommand{\ceil}[1]{\left\lceil{#1}\right\rceil}
\newcommand{\floor}[1]{\left\lfloor{#1}\right\rfloor}
\newcommand{\card}[1]{\left|{#1}\right|}
\newcommand{\setcomp}[1]{\overline{#1}}
\newcommand{\provided}{\,:\,}
\newcommand{\divs}{\mid}
\newcommand{\ndivs}{\nmid}
\newcommand{\iequiv}[1]{\,\overset{#1}{\equiv}\,}
\newcommand{\imod}[1]{\allowbreak\mkern5mu(#1\,\,\text{پیمانه‌ی})}

\newcommand{\poly}{\mathop{\mathrm{poly}}}
\newcommand{\polylog}{\mathop{\mathrm{polylog}}}
\newcommand{\eps}{\varepsilon}

\newcommand{\lee}{\leqslant}
\newcommand{\gee}{\geqslant}
\renewcommand{\leq}{\lee}
\renewcommand{\le}{\lee}
\renewcommand{\geq}{\gee}
\renewcommand{\ge}{\gee}

\newcommand{\مهم}[1]{\textbf{#1}}
\renewcommand{\برچسب}{\label}

\newcommand{\REM}[1]{}
\renewcommand{\حذف}{\REM}
\newcommand{\لر}{\lr}
\newcommand{\کد}[1]{\lr{\tt #1}}
\newcommand{\پاورقی}[1]{\footnote{\lr{#1}}}



% -------------------- Dictionary --------------------


\newcommand{\dicalphabet}[1]{
	\begin{minipage}{\columnwidth}
		\centerline{\noindent\textbf{\large #1 }}
		\vspace{.5em}
	\end{minipage}
	\nopagebreak[4]
}

\newcommand{\dic}[2]{\noindent  #2 \dotfill  \lr{#1} \\ }


% ------------------------------ Images and Figures --------------------------

\graphicspath{{figs/}}
\setlength{\intextsep}{0pt}  % for float boxes
\renewcommand{\psscalebox}[1]{}  % for LaTeX Draw

\newcommand{\floatbox}[2]
{\begin{wrapfigure}{l}{#1}
		\centering #2 \end{wrapfigure}}

\newcommand{\centerfig}[2]
{\centering\scalebox{#2}{\input{figs/#1}}}

\newcommand{\fig}[3]
{\floatbox{#3}{\centerfig{#1}{#2}}}

\newcommand{\centerimg}[2]
{\vspace{1em}\begin{center}\includegraphics[width=#2]{figs/#1}\end{center}\vspace{-1.5em}}

\NewDocumentCommand{\img}{m m o}
{\begin{wrapfigure}{l}{\IfValueTF{#3}{#3}{#2}}
		\centering\includegraphics[width=#2]{figs/#1}\end{wrapfigure}}




\newcommand{\glossaryTitle}[1]{
	\hline
	\hline
	\multicolumn{2}{|c|}{}  \\
	\multicolumn{2}{|c|}{\linebreak \LARGE\textbf{{#1}}}  \\
	\multicolumn{2}{|c|}{}  \\ \hline \hline
	
}

\newcommand{\glossaryEntry}[4]{
	#1
	&
	\textbf{معنا} \hspace{1.53cm} #2
	
	\textbf{مترادف} \hspace{1cm} #3
	
	\textbf{متشابه}\hspace{1.28cm} #4
	\\
	\hline
	
}

\newcommand{\calendarEntry}[6]{
		#1 & #2 & $#3$ & ۱۴۰۱/#4 & ۱۴۰۱/#5 & #6\\
		\hline
}


\newcommand{\changelogEntry}[2]{
	 ۱۴۰۱/#1 & #2\\
	\hline
}

