\chapter{ارزیابی دستاوردهای تکرار}
در این بخش محصولات تکرار با استفاده از تعدادی معیار سنجیده می‌شوند. این معیارها با توجه به محصولات هر تکرار، به‌روزرسانی و تکمیل خواهند شد.
\section{چک‌لیست‌های ارزیابی}

\iffalse
Item icons for checklists:	
	\item[$\square$]
	\item[$\boxtimes$]
\fi

\subsection{اهداف کلی فاز تفصیل}
در این قسمت دستاوردهایی که باید در فاز تفصیل تحقق یابد، بررسی می‌شود. از آنجا که این اهداف باید در طول سه تکرار محقق شوند، این لیست به‌مرور تکمیل خواهد شد:
\begin{itemize} % \setlength\itemsep{0.1cm}
	\item[$\boxtimes$]
	ایجاد نرم‌افزار اجرایی پایه‌ی معماری\LTRfootnote{Executable architectural baseline}
	\item[$\square$]
	تکمیل نرم‌افزار اجرایی
	\item[$\boxtimes$]
	تعریف حدود 80 درصد از نیازمندی‌های وظیفه‌ای سیستم
	\item[$\square$]
	تعریف شاخصه‌های کیفیت (مانند نرخ یافتن نقص)
	\item[$\square$]
	ایجاد یک برنامه با جزئیات برای فاز construction
	
\end{itemize}

\subsection{نکات کلی جریان‌های کاری نیازمندی و تحلیل}

\begin{itemize}
	\item[$\boxtimes$]
	تصحیح مرز زیرسیستم‌ها با توجه به اصلاح نیازمندی‌ها در تکرار دوم
	\item[$\square$]
	تکرار فرآیند مدل‌سازی موارد کاربرد جهت یافتن موارد کاربرد و کنش‌گرهای جدید و پایدار شدن مرزها
	\item[$\boxtimes$]
	رعایت سادگی در مدل‌سازی تحلیل
	\item[$\boxtimes$]
	صرفاً کلاس‌هایی در مدل‌سازی معرفی شوند که اجزای قلمرو مسئله را بیان می‌کنند.
	\item[$\boxtimes$]
	هرگز تصمیمی با در نظر داشتن جنبه‌ی پیاده‌سازی گرفته نشده باشد.
	\item[$\boxtimes$]
	با دقت در طراحی کلاس‌ها و روابط، \lr{Coupling} کمینه شود.
	\item[$\boxtimes$]
	از رابطه‌ی وراثت\LTRfootnote{\lr{Inheritence}} تنها در صورت وجود سلسله‌مراتب در مفاهیم استفاده شود و از استفاده‌ی زاید آن پرهیز شده باشد.
	\item[$\square$]
	واژگان متشابه و مترادف در مدل تحلیل بررسی شده باشند.
	\item[$\boxtimes$]
	نمودارهای لازم برای تحقق موارد کاربرد (نمودار کلاس تحلیل و نمودارهای تعاملی) رسم شده باشند.
	\item[$\square$]
	نیازمندی‌ها و موارد کاربرد جدیدی که هنگام محقق‌سازی موارد کاربرد پیدا می‌شوند ضبط و مستند شده باشند و تغییرهای لازم در موارد کاربرد پیش از محقق‌سازی آن‌ها اعمال شده باشد.
\end{itemize}

\subsection{کارت‌های \lr{CRC}}

\begin{itemize}
	\item[$\boxtimes$]
	تفکیک مسئولیت‌های مختلف هر کلاس در سطرهای جداگانه
	\item[$\boxtimes$]
	نام کلاس نشان‌گر منظور آن باشد و به یک وجهه از قلمرو مسئله نگاشت شود.
	\item[$\boxtimes$]
	لیست مسئولیت‌ها کوتاه و خوش‌تعریف باشد.
	\item[$\boxtimes$]
	ویژگی‌های کلاس انسجام\LTRfootnote{\lr{Cohesion}} بالایی داشته باشند.
	\item[$\boxtimes$]
	میزان جفت‌شدگی\LTRfootnote{\lr{Coupling}} با سایر کلاس‌ها پایین باشد (برای تحقق منظور خود نیاز به ارتباط با تعداد کمی کلاس دیگر داشته باشد).
	\item[$\boxtimes$]
	کلاس‌های بسیار کوچک زیاد یا صرفاً چند کلاس بزرگ وجود نداشته باشد.
	\item[$\boxtimes$]
	هیچ کلاسی قادر مطلق نباشد و درخت وراثت عمیقی نیز نداشته باشد.
	\item[$\boxtimes$]
	داشتن کلاس‌های کاتالوگ
\end{itemize}

\subsection{نمودارهای موارد کاربرد و فعالیت و توالی}

\begin{itemize}
	\item[$\boxtimes$]
	ساده و کوتاه نگه داشتن نمودارهای موارد کاربرد
	\item[$\boxtimes$]
	تمرکز روی نیاز کنش‌گر از سیستم به‌جای چگونگی انجام در نمودار موارد کاربرد
	\item[$\boxtimes$]
	خودداری از \lr{Functional Decomposition} در نمودار موارد کاربرد
	\item[$\boxtimes$]
	کاربرد صحیح \lr{Notation} نمودار فعالیت (نمادها، سیگنال و رویداد و ...)
	\item[$\boxtimes$]
	استفاده صحیح از \lr{Swim Lane} در نمودار فعالیت
	\item[$\boxtimes$]
	در نمودارهای توالی جهت صحیح حرکت زمان و \lr{lifeline}ها رعایت شده باشد.
	\item[$\boxtimes$]
	به‌طور کلی علائم و عبارات (\lr{notation}) نمودار توالی (شامل خط‌چین‌ها، قیدها و ...) رعایت شده باشد.
	\item[$\boxtimes$]
	از امکانات ارائه شده توسط \lr{Combined fragment}ها به‌خوبی استفاده شده باشد.
	
\end{itemize}

\subsection{نمودارهای بسته و کلاس تحلیل}

\begin{itemize}
	\item[$\boxtimes$]
	رابطه \lr{Association} تنها در صورتی میان دو شیء تعریف شده باشد که میان کلاس‌های آن دو شیء یک ارتباط یا وابستگی وجود داشته باشد.
	\item[$\boxtimes$]
	المان‌های مرتبط با رابطه \lr{Association} اعم از نام، نقش، چندی و مسیر رابطه به‌درستی رسم شده باشد. ارجحیت روابط نیز مدنظر قرار گرفته شده باشد.
	\item[$\boxtimes$]
	بسته‌ها به‌درستی المان‌های مرتبط از نظر معنا را گروه‌بندی کرده باشد.
	\item[$\boxtimes$]
	بسته‌های تحلیل یک سلسله‌مراتب تشکیل دهند و با توجه به مفاهیم موجود در خود لایه‌بندی شده باشد.
	\item[$\boxtimes$]
	هر بسته در نمودارها یک فضای نام کپسوله شده با نام‌های یکتا باشد.
	\item[$\boxtimes$]
	میزان جفت‌شدگی (\lr{Coupling}) بین بسته‌ها با تنظیم صحیح میزان نمایان بودن\LTRfootnote{\lr{Visiblity}} کنترل شده باشد؛ یعنی تلاش شده باشد که وابستگی‌های بین بسته‌ها کمینه شود، تعداد اعضای \lr{public} تا جای ممکن کم و تعداد المان‌های \lr{private} تا حد امکان زیاد شود.
	\item[$\square$]
	از \lr{stereotype}ها استفاده‌ شده باشد.
	\item[$\boxtimes$]
	روابط وابستگی\LTRfootnote{\lr{Dependency}} میان بسته‌ها با نوع رابطه‌ی وابستگی مشخص شده باشد.

\end{itemize}

\subsection{نرم‌افزار اجرایی پایه معماری}
با توجه به زمان‌بندی و تکمیل این بخش در تکرار دوم فاز تفصیل، برخی از معیارها در تکرار بعد اضافه و تکمیل می‌شوند.
\begin{itemize}
	\item[$\square$]
	عملیات عمومی کلاس‌ها با کارخواهان یک قرارداد تعریف کند.
	\item[$\boxtimes$]
	تمامیت\LTRfootnote{\lr{Completeness}} (کاربرد کلاس کم‌تر از انتظار معقول کارخواه نباشد)
	\item[$\boxtimes$]
	کفایت\LTRfootnote{\lr{Sufficiency}} (کاربرد کلاس بیشتر از انتظار معقول کارخواه نباشد)
	\item[$\boxtimes$]
	بدوی بودن\LTRfootnote{\lr{Primitveness}} (خدمات کلاس ساده، اتمیک و منحصربه‌فرد باشد)
	\item[$\boxtimes$]
	انسجام بالا به معنی مستقل و واحد بودن مفهوم هر کلاس و پشتیبانی اعمال کلاس از منظور آن
	\item[$\boxtimes$]
	جفت‌شدگی پایین به معنی خودداری از اتصال دو کلاس بدون ازتباط معنایی صحیح و یا به جهت استفاده مجدد از کد و همچنین به معنی محدود بودن اتصالات کلاس با سایرین در حدی که منظور آن کلاس را تحقق بخشد.
	\item[$\boxtimes$]
	ظاهر مناسب واسط کاربری و پیاده‌سازی نیمی از نیازمندی‌ها در نمونه‌ی اولیه
\end{itemize}
