\chapter{ارزیابی دستاوردهای تکرار}
در این بخش محصولات تکرار با استفاده از تعدادی معیار سنجیده می‌شوند. این معیارها با توجه به محصولات هر تکرار، به‌روزرسانی و تکمیل خواهند شد.
\section{چک‌لیست‌های ارزیابی}

\iffalse
Item icons for checklists:	
	\item[$\square$]
	\item[$\boxtimes$]
\fi

\subsection{\hspace*{0.2cm}اهداف کلی فاز \lr{Construction}}
در این قسمت دستاوردهایی که باید در فاز \lr{Construction} تحقق یابد، بررسی می‌شود.
\begin{itemize} \setlength\itemsep{0cm}
	\item[$\square$]
	تمام نیازمندی‌ها، تحلیل و طراحی سیستم کامل شود.
	\item[$\square$]
	نرم‌افزار اجرایی اولیه‌ای که در فاز تفصیل ایجاد شد تبدیل به سیستم نهایی شود.
	\item[$\boxtimes$]
	در جریان کاری نیازمندی‌ها تمام نیازمندی‌هایی که قبلاً دیده نشدند (حدود 20 درصد) شناسایی شوند.
	\item[$\boxtimes$]
	در جریان‌های کاری تحلیل و طراحی، مدل تحلیل و مدل طراحی نهایی شود.
	\item[$\square$]
	در جریان کاری پیاده‌سازی و تست به‌طور کلی ظرفیت عملیاتی اولیه\LTRfootnote{\lr{Initial Operational Capability}} ساخته و تست شود.
	\item[$\square$]
	مستندات نهایی سیستم اعم از مستند نصب و استفاده تهیه شود.
	
\end{itemize}

\subsection{\hspace*{0.2cm}نکات کلی جریان کاری \lr{Implementation}}
این جریان کاری، مهم‌ترین جریان در این فاز است. مواردی که برای این جریان کاری باید ارزیابی شود عبارتند از:
\begin{itemize} \setlength\itemsep{0cm}
	\item[$\square$]
	نمودار استقرار\LTRfootnote{\lr{Deployment Diagram}} سیستم تهیه شود.
	\item[$\square$]
	نمودار استقرار بتواند به‌درستی معماری نرم‌افزار را به سخت‌افزار نگاشت کند.
	\item[$\square$]
	از \lr{Sterotype} ها و \lr{Descriptor} ها در نمودار استقرار به‌درستی استفاده شود.
	\item[$\square$]
	تمامی \lr{Artifact} ها ساخته و روی \lr{Node} (ها) مستقر شوند.
	\item[$\square$]
	مدل طراحی کاملاً به کد قابل اجرا تبدیل شود.
	\item[$\square$]
	پیاده‌سازی تست‌های واحد و تست‌های جامع

\end{itemize}

\subsection{\hspace*{0.2cm}کارت‌های \lr{CRC} (تکمیل شده)}

\begin{itemize} \setlength\itemsep{0cm}
	\item[$\boxtimes$]
	تفکیک مسئولیت‌های مختلف هر کلاس در سطرهای جداگانه
	\item[$\boxtimes$]
	نام کلاس نشان‌گر منظور آن باشد و به یک وجهه از قلمرو مسئله نگاشت شود.
	\item[$\boxtimes$]
	لیست مسئولیت‌ها کوتاه و خوش‌تعریف باشد.
	\item[$\boxtimes$]
	ویژگی‌های کلاس انسجام\LTRfootnote{\lr{Cohesion}} بالایی داشته باشند.
	\item[$\boxtimes$]
	میزان جفت‌شدگی\LTRfootnote{\lr{Coupling}} با سایر کلاس‌ها پایین باشد (برای تحقق منظور خود نیاز به ارتباط با تعداد کمی کلاس دیگر داشته باشد).
	\item[$\boxtimes$]
	کلاس‌های بسیار کوچک زیاد یا صرفاً چند کلاس بزرگ وجود نداشته باشد.
	\item[$\boxtimes$]
	هیچ کلاسی قادر مطلق نباشد و درخت وراثت عمیقی نیز نداشته باشد.
	\item[$\boxtimes$]
	داشتن کلاس‌های کاتالوگ
\end{itemize}

\subsection{\hspace*{0.2cm}نمودارهای موارد کاربرد و فعالیت و توالی}

\begin{itemize} \setlength\itemsep{0cm}
	\item[$\boxtimes$]
	ساده و کوتاه نگه داشتن نمودارهای موارد کاربرد
	\item[$\boxtimes$]
	تمرکز روی نیاز کنش‌گر از سیستم به‌جای چگونگی انجام در نمودار موارد کاربرد
	\item[$\boxtimes$]
	خودداری از \lr{Functional Decomposition} در نمودار موارد کاربرد
	\item[$\boxtimes$]
	کاربرد صحیح \lr{Notation} نمودار فعالیت (نمادها، سیگنال و رویداد و ...)
	\item[$\boxtimes$]
	استفاده صحیح از \lr{Swim Lane} در نمودار فعالیت
	\item[$\boxtimes$]
	در نمودارهای توالی جهت صحیح حرکت زمان و \lr{lifeline}ها رعایت شده باشد.
	\item[$\boxtimes$]
	به‌طور کلی علائم و عبارات (\lr{notation}) نمودار توالی (شامل خط‌چین‌ها، قیدها و ...) رعایت شده باشد.
	\item[$\boxtimes$]
	از امکانات ارائه شده توسط \lr{Combined fragment}ها به‌خوبی استفاده شده باشد.
	
\end{itemize}

\subsection{\hspace*{0.2cm}نمودارهای بسته و کلاس تحلیل}

\begin{itemize} \setlength\itemsep{0cm}
	\item[$\boxtimes$]
	رابطه \lr{Association} تنها در صورتی میان دو شیء تعریف شده باشد که میان کلاس‌های آن دو شیء یک ارتباط یا وابستگی وجود داشته باشد.
	\item[$\boxtimes$]
	المان‌های مرتبط با رابطه \lr{Association} اعم از نام، نقش، چندی و مسیر رابطه به‌درستی رسم شده باشد. ارجحیت روابط نیز مدنظر قرار گرفته شده باشد.
	\item[$\boxtimes$]
	بسته‌ها به‌درستی المان‌های مرتبط از نظر معنا را گروه‌بندی کرده باشد.
	\item[$\boxtimes$]
	بسته‌های تحلیل یک سلسله‌مراتب تشکیل دهند و با توجه به مفاهیم موجود در خود لایه‌بندی شده باشد.
	\item[$\boxtimes$]
	هر بسته در نمودارها یک فضای نام کپسوله شده با نام‌های یکتا باشد.
	\item[$\boxtimes$]
	میزان جفت‌شدگی (\lr{Coupling}) بین بسته‌ها با تنظیم صحیح میزان نمایان بودن\LTRfootnote{\lr{Visiblity}} کنترل شده باشد؛ یعنی تلاش شده باشد که وابستگی‌های بین بسته‌ها کمینه شود، تعداد اعضای \lr{public} تا جای ممکن کم و تعداد المان‌های \lr{private} تا حد امکان زیاد شود.
	\item[$\boxtimes$]
	از \lr{stereotype}ها استفاده‌ شده باشد.
	\item[$\boxtimes$]
	روابط وابستگی\LTRfootnote{\lr{Dependency}} میان بسته‌ها با نوع رابطه‌ی وابستگی مشخص شده باشد.
	\item[$\boxtimes$]
	استفاده از الگوهای مفید طراحی در بخش‌های مناسب کد مورد توجه قرار گرفته باشد.

\end{itemize}

\subsection{\hspace*{0.2cm}کلاس‌های طراحی}

\begin{itemize} \setlength\itemsep{0cm}
	\item[$\boxtimes$]
	تمامی کلاس‌های تحلیل، حداقل به یک جزء از مدل طراحی نظیر شوند.
	\item[$\boxtimes$]
	تمامی منابع کلاس‌های طراحی (کلاس‌های تحلیل با جزئیات بیشتر در قلمرو مسئله و کلاس‌های کتاب‌خانه‌های ابزاری، میان‌افزارها، نرم‌افزارهای \lr{COTS} در صورت وجود، کتاب‌خانه‌های \lr{GUI} در قلمرو جواب و همچنین کلاس‌هایی که به‌واسطه‌ی استفاده از الگوهای طراحی در پیاده‌سازی دیده می‌شوند) در نظر گرفته شوند.
	\item[$\boxtimes$]
	کلاس‌های طراحی -برخلاف کلاس‌های تحلیل- دارای جزئیات کامل (شامل نام و نوع مقدار اولیه و سطح دسترسی برای ویژگی‌‌ها و توابع کلاس و همچنین پارامترهای اختیاری و نوع مقدار خروجی برای توابع کلاس) باشند.
	\item[$\boxtimes$]
	تمام ویژگی‌های خوش‌فرم بودن\LTRfootnote{\lr{Well-formedness}} (جامع و مانع بودن و ساده و اتمیک بودن و انسجام بالا و جفت‌شدگی پایین) در کلاس‌های طراحی نیز مورد توجه جدی قرار گرفته باشند.
	\item[$\boxtimes$]
	از \lr{Inheritence} تنها جاهایی استفاده شود که یک مفهوم \textit{\lr{is-a}} وجود داشته باشد.
	\item[$\boxtimes$]
	از \lr{Multiple Inheritence} مگر در \lr{Mixin}ها ترجیحاً اجتناب شود.
	\item[$\boxtimes$]
	روابطی که در تحلیل قبلاً تعریف شده بود با اضافه کردن جزئیاتی همچون جهت، چندی، نقش و نوع ارتباط برای پیاده‌سازی کاملاً آماده شوند.
	\item[$\boxtimes$]
	به تفاوت رابطه‌ی \lr{Composition} با رابطه‌ی \lr{Aggregation} عادی توجه شود.
	\item[$\square$]
	در صورت لزوم برای مدل‌سازی کلاس‌های پیچیده با ساختار داخلی مرکب از نمودار \lr{Composite Structure} استفاده شود.
\end{itemize}

\subsection{\hspace*{0.2cm}نمودارهای مؤلفه و توالی طراحی}

\begin{itemize} \setlength\itemsep{0cm}
	\item[$\boxtimes$]
	سیستم تا جای ممکن به زیرسیستم‌های مستقل شکسته شده باشد. این زیرسیستم‌ها نشان‌گر مؤلفه‌های درشت‌دانه و مقاصد مجزای طراحی باشند.
	\item[$\boxtimes$]
	تعامل بین زیرسیستم‌ها تنها از طریق \lr{Interface} محقق شود.
	\item[$\boxtimes$]
	برای تعیین پروتکل‌های مشترک دو کلاس که رابطه توارث ندارند از \lr{Interface}ها استفاده شده باشد.
	\item[$\boxtimes$]
	برای مؤلفه‌های سیستم، عملیات و ویژگی‌ها، روابط و ساختار درونی، و \lr{Interface} های فراهم شده یا مورد استفاده مشخص شود.
	\item[$\boxtimes$]
	در نمایش مؤلفه‌ها از \lr{stereotype} ها به‌درستی استفاده شده باشد.
	\item[$\boxtimes$]
	برای مخفی کردن جزئیات زیرسیستم‌ها در صورت لزوم از الگوهایی مانند \lr{Facade} و \lr{layering} استفاده شود.
	\item[$\boxtimes$]
	تحقق موارد کاربرد طراحی با گسترش تحقق‌های موارد کاربرد تحلیل ایجاد شوند و به تحقق متناظر خود \lr{trace} شوند.
	\item[$\boxtimes$]
	برای نمایش هم‌روندی در نمودار توالی طراحی از قطعات ترکیبی \lr{par} و \lr{critical} به‌درستی استفاده شود.
	
\end{itemize}

\subsection{\hspace*{0.2cm}اصول طراحی شیءگرا}
نیاز به نگه‌داشت‌پذیری سیستم، قابل اتکا بودن سیستم و امکان ردیابی کلاس‌های طراحی در طول پروژه تنها گوشه‌ای از موارد هستند که ایجاب می‌کنند که اصول طراحی شیءگرا در ایجاد نرم‌افزار به‌طور جدی در دستور کار قرار گیرد.

\begin{itemize} \setlength\itemsep{0cm}
	\item[$\boxtimes$]
	کلاس‌ها باید برای افزودن ویژگی باز، اما برای ویرایش بسته باشند و گسترش نرم‌افزار با حداقل نیاز به تغییر کد موجود امکان‌پذیر باشد (\lr{OCP}).
	\item[$\boxtimes$]
	لازم است \lr{Coupling} موجود در نرم‌افزار در سطح \lr{Abstract} باشد نه کلاس‌های صلب (\lr{DIP}).
	\item[$\boxtimes$]
	تمام واسط‌ها\LTRfootnote{\lr{Interface}} به‌صورت ریزدانه و با انسجام بالا تعریف شده باشند (\lr{ISP}).
	\item[$\boxtimes$]
	از رابطه‌ی وراثت صرفاً به‌منظور استفاده‌ی مجدد از کد استفاده نشود و برای این مقصود از \lr{Delegation} استفاده شود (\lr{CRP}).
	\item[$\boxtimes$]
	از به‌وجود آمدن دید ترایا جلوگیری شود و با محدود کردن امکان فراخوانی عملیات یک کلاس به خود، پارامترها و اشیاء درونی، قاعده‌ی \lr{PLK} رعایت شود.
	\item[$\square$]
	اصول \lr{GRASP} در طراحی مدنظر قرار گیرند.
\end{itemize}
%

\subsection{\hspace*{0.2cm}نرم‌افزار اجرایی پایه معماری (\lr{EAB})}
با توجه به زمان‌بندی و تکمیل این بخش در تکرار دوم فاز تفصیل، برخی از معیارها در تکرار بعد اضافه و تکمیل می‌شوند.

\begin{itemize} \setlength\itemsep{0cm}
	\item[$\boxtimes$]
	ویژگی‌های مهم از نظر معماری -اعم از وظیفه‌ای و غیروظیفه‌ای- در نرم‌افزار اولیه در پایان این فاز پیاده‌سازی شده باشند.
	\item[$\boxtimes$]
	عملیات عمومی کلاس‌ها با کارخواهان یک قرارداد تعریف کند.
	\item[$\boxtimes$]
	تمامیت\LTRfootnote{\lr{Completeness}} (کاربرد کلاس کم‌تر از انتظار معقول کارخواه نباشد)
	\item[$\boxtimes$]
	کفایت\LTRfootnote{\lr{Sufficiency}} (کاربرد کلاس بیشتر از انتظار معقول کارخواه نباشد)
	\item[$\boxtimes$]
	بدوی بودن\LTRfootnote{\lr{Primitveness}} (خدمات کلاس ساده، اتمیک و منحصربه‌فرد باشد)
	\item[$\boxtimes$]
	انسجام بالا به معنی مستقل و واحد بودن مفهوم هر کلاس و پشتیبانی اعمال کلاس از منظور آن
	\item[$\boxtimes$]
	جفت‌شدگی پایین به معنی خودداری از اتصال دو کلاس بدون ازتباط معنایی صحیح و یا به جهت استفاده مجدد از کد و همچنین به معنی محدود بودن اتصالات کلاس با سایرین در حدی که منظور آن کلاس را تحقق بخشد.
	\item[$\boxtimes$]
	ظاهر مناسب واسط کاربری و پیاده‌سازی حدوداً نیمی از نیازمندی‌ها در نمونه‌ی اولیه
\end{itemize}
