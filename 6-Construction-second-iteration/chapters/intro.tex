
\chapter{مقدمات}

\section{شرح سیستم آرنو}

سیستم مورد بررسی در این پروژه، ‌یک سیستم جامع اطلاعاتی برای مدیریت ارائه‌ی خدمات در خانه است. 
در این مسئله امکان ارائه‌ی خدمات در سه حوزه‌ی زیر مورد اهمیت است:

\begin{enumerate}
	\item
	نظافت: شامل نظافت بخش‌های مختلف منزل 
	\item
	حمل و نقل: شامل خدمات انتقال بسته و اسباب‌کشی
	\item 
	کارهای فنی: شامل خدمات لوله‌کشی و تاسیسات، برق‌کاری، تعمیرات لوازم خانگی، تعمیر رایانه و موبایل و تعمیر و سرویس خودرو
	
\end{enumerate}

افراد در تعامل با این سامانه، از یک سو مشتریانی بوده که درخواست ارائه‌ی خدمات را در سیستم ثبت می‌کنند و از سوی دیگر متخصصانی هستند که هر کدام می‌توانند در حوزه‌ی تخصصی خود به ارائه‌ی خدمت به مشتریان بپردازند. نوع خدمات ارائه شده در شرکت و توسط متخصصین مختلف می‌تواند در طول زمان دچار تغییر شود و این تغییرات در این سامانه پیش‌بینی شده است. همچنین یک ویژگی مورد توجه دیگر در این سامانه امکان ارزیابی متخصصین و مشتریان توسط یک‌دیگر است تا امکان شناسایی نقاط ضعف و قوت ارائه‌ی خدمات وجود داشته باشد.

به غیر از کاربران عادی، مخاطبان دیگر این سامانه مدیران شرکت هستند. مدیران کسب‌وکاری نیاز به تعامل با این سیستم جهت گرفتن گزارشات آماری و تفصیلی دارند تا روند ارائه‌ی خدمات را بررسی کند و برای جهت کسب‌و‌کار تصمیم بگیرند. مدیران فنی نیز گزارشات مشکلات فنی سیستم را دریافت و بررسی می‌کنند تا مشکلات پیش آمده را برطرف و نرخ خطاهای سیستم را کاهش دهند و درنتیجه رضایت‌مندی کاربران در استفاده از سیستم را فراهم کنند.


\section{شرح مستندات}
\subsection{نیازمندی‌ها}
در اولین اجرا از جریان کاری تحلیل نیازمندی‌ها، نیازمندی‌های وظیفه‌ای استخراج و براساس قواعد \lr{MoSCoW} و ریسک و هزینه‌ی پیاده‌سازی اولویت‌بندی و گزارش شده‌اند. همچنین نیازمندی‌های غیروظیفه‌ای ناظر به تمامی موارد کاربرد سیستم در چند دسته ارائه شده‌اند.
\subsection{ریسک‌ها}
ریسک‌هایی که ممکن است سر راه ایجاد نرم‌افزار در پروژه قرار گیرند از دید مهندسی محصول، محیط ایجاد و محدودیت‌های برنامه بررسی شده‌اند. همچنین در راستای مدیریت ریسک‌ها سعی شده راه‌حل‌هایی برای هر یک از موارد ارائه شود.
\subsection{موارد کاربرد}
مستندات موارد کاربرد شامل توصیف کنش‌گرها، نمودارها و توصیف‌های موارد کاربرد بخش اصلی این مجموعه مستندات را تشکیل می‌دهند. در این مستندات با تقسیم موارد کاربرد سیستم به چهار زیر سیستم کاربری، خدمت‌دهی، بازخورد و گزارش‌گیری هریک از نیازمندی‌های وظیفه‌ای واجد اهمیت بیشتر طبق اولویت‌بندی در جداول جداگانه بررسی شده‌اند.
\subsection{واژه‌نامه}
در پیوست، واژگان کلیدی به‌کار رفته در مستندات به‌همراه واژگان مترادف و متشابه آن‌ها با هدف ایجاد یک درک مشترک بین تیم ایجاد و مشتریان شرح داده شده است.

\section{بررسی و استفاده}
سیستم آرنو برای بررسی و استفاده بر روی آدرس \href{http://arno.mostafaojaghi.ir}{\lr{arno.mostafaojaghi.ir}} قابل دسترسی است.
روی این سیستم سه مشتری با نام‌های کاربری \lr{c1, c2, c3}، سه متخصص با نام‌های کاربری \lr{s1, s2, s3} و سه مدیر با نام‌های کاربری \lr{a1, a2, a3} ایجاد شده اند.
مدیر اول و دوم از نوع مدیر شرکت و مدیر سوم از نوع مدیر فنی است.
رمز عبور همه‌ی این حساب‌ها \lr{Q8RX7psb5zyw} قرار داده شده‌است.
همچنین یک ادمین در سطح \lr{django superuser} وجود دارد که از طریق \href{http://arno.mostafaojaghi.ir/admin}{\lr{arno.mostafaojaghi.ir/admin}} با نام کاربری \lr{admin} و رمز عبور \lr{3A<ToRaC*c|VgLx-} در دسترس است.