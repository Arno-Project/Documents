\chapter{راهنمای نصب}

آرنو برای سهولت در نصب و راه‌اندازی و استقلال از سکوی مورد استفاده از \lr{Docker} و \lr{Docker Compose} استفاده می‌کند.
بنابراین پیش از اجرای آن باید این دو پیش‌نیاز به ترتیب زیر نصب شوند. پس از آن طبق توضیحات می‌توان آرنو را اجرا کرد.
توصیه‌ی ما نصب و اجرای آرنو بر روی یک سرور لینوکسی است اما می‌توان برای امتحان کردن محصول، آن را برروی رایانه‌ی شخصی نیز نصب و استفاده کرد.
همچنین برای یکپارچه‌سازی دستورات اجرا از make استفاده شده است که می‌توانید آن را به راحتی با کمک مدیر بسته‌ی سیستم‌عامل خود (apt برای debian, choco برای Windows و brew برای macOS) نصب کنید.

\section{نصب Docker}

برای نصب Docker روی سیستم‌های لینوکسی مطابق توضیحات داده‌شده در \href{https://docs.docker.com/compose/install/compose-plugin/#install-using-the-repository}{این لینک} عمل کنید. 
همچنین برای نصب آن روی رایانه‌های شخصی می‌توانید براساس سیستم‌عامل خود از راهنمای موجود در \href{https://docs.docker.com/get-docker/}{این صفحه} استفاده کنید.

\section{نصب \lr{Docker Compose}}

اگر برای نصب داکر از روشی غیر از نصب \lr{Docker Desktop} استفاده کرده‌اید لازم است تا به صورت جداگانه \lr{Docker Compose} را با استفاده از دستور زیر نصب کنید.
\lstset{language=Bash}
\lstset{frame=lines}
\lstset{caption={Installing Docker Compose}}
\lstset{label={lst:code_direct}}
\begin{latin}
\begin{lstlisting}
sudo apt install docker-compose
\end{lstlisting}
\end{latin}

\section{اجرای آرنو}

برای اجرای آرنو باید ابتدا کدهای آن را از دو مخزن
\href{https://github.com/Arno-Project/Frontend}{Frontend} و
\href{https://github.com/Arno-Project/Backend}{Backend}
دریافت کنید.
در پوشه‌ی backend فایل \lr{.env.template} را کپی کرده و با نام \lr{.env} ذخیره کرده و مقادیر خالی را درون آن پر کنید.
سپس در هر مخزن به شاخه‌ای که Makefile در آن قرار دارد رفته و دستور \lr{make build} را اجرا کنید.
این دستور پس از دریافت و ساخت \lr{docker image} های مورد نیاز، آن‌ها را در پس زمینه اجرا می‌کند.

برای توقف اجرای برنامه در همان شاخه از دستور \lr{make down} و برای اجرای مجدد آن از \lr{make up} استفاده کنید.
